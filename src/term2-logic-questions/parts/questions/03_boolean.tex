\Section{Булева логика}{Игорь Энгель + Чистякова Полина}
\Subsection{Билет 18: <<Высказывания и операции. Тавтологии.>>}
\begin{definition} \thmslashn 

    Определим множество <<пропозициональных формул>> (высказывинй) следующим обарзом:

    \begin{itemize}
        \item <<пропозициональная переменная>> является высказыванием
        \item Если $A$ - высказывание, то $\neg A$ (НЕ $A$) - высказывание.
        \item Если $A$ и $B$ - высказывания, то $A \land B$ ($A$ И $B$), $A \lor B$ ($A$ ИЛИ $B$), $A \to B$ (из $A$ следует $B$) - высказывания.
    \end{itemize}
\end{definition}
\begin{definition} \thmslashn 

    Пусть высказывание $A$ содержит пропозиональные переменные $x_1, \ldots, x_{n}$.

    Соответствующий высказыванию булевой функцией называется функция $\phi_{A} : \mathbb{B}^{n} \mapsto \mathbb{B}$, где $\mathbb{B} = \{0, 1\} = \mathbb{Z}/2$, заданная индуктивно следуюзим образом:

    \begin{equation*}
        \begin{array}{|c|c|c|c|c|c|} \hline
            A & B & A \land B & A \lor B & A \to B & \neg A \\ \hline
            1 & 1 & 1 & 1 & 0 & 0\\ \hline
            1 & 0 & 0 & 1 & 0 & 0 \\ \hline
            0 & 1 & 0 & 1 & 1 & 1 \\ \hline
            0 & 0 & 0 & 0 & 1 & 1\\ \hline
        \end{array}
    \end{equation*}


\end{definition}

\begin{definition} \thmslashn 
    
    Тавтологией называется высказывание, соответветсвующия которому формула принимает значение $1$ на всех возможных входах. 

\end{definition}

\begin{definition} $a \leftrightarrow b = (a \to b) \land (b \to  a)$ \end{definition}

\begin{example}[Примеры тавтологий] \thmslashn

    \begin{enumerate}
        \item $(p \land q) \leftrightarrow (q \land p)$
        \item $(p \lor q) \leftrightarrow (q \lor p)$
        \item $((p \land q) \land r) \leftrightarrow (p \land (q \land r))$
        \item $((p \lor q) \lor r) \leftrightarrow (p \lor (q \lor r))$
        \item $(p \land (q \lor r)) \leftrightarrow ((p \land q) \lor (p \land r))$
        \item $(p \lor (q \land r)) \leftrightarrow ((p \lor q) \land (p \lor r))$
        \item $\neg(p \land q) \leftrightarrow (\neg p \lor \neg q)$
        \item $\neg(p\lor q) \leftrightarrow (\neg p \land \neg q)$
        \item $(p \lor (p \land q)) \leftrightarrow p$
        \item $(p \land (p \lor q)) \leftrightarrow p$
        \item $(p \to q) \leftrightarrow (\neg q \to \neg p)$
        \item $p \leftrightarrow \neg\neg p$
    \end{enumerate}
\end{example}

\begin{example}[Тоже самое в другой нотации] \thmslashn

    А то об те убиться можно\ldots

    \begin{enumerate}
        \item $pq = qp$
        \item $p+q = q+p$
        \item $(pq)r = p(qr)$
        \item $(p+q)+r = p+(q+r)$
        \item $p(q+r) = pq+pr$
        \item $p+qr = (p+q)(p+r)$
        \item $\overline{pq} = \overline{p} + \overline{q}$ 
        \item $\overline{p + q} = \overline{p}\overline{q}$ 
        \item $p + pq = p$
        \item $p(p+q) = p$
        \item $p \to q = \overline{q} \to \overline{p}$ 
        \item $p = \overline{\overline{p}}$
    \end{enumerate}
\end{example}

\Subsection{Билет 19: <<Выразимость любой формулы в КНФ и ДНФ>>}

\begin{definition} \thmslashn 

    Формула находится в конъюнктивной нормальной форме (КНФ) если она имеет вид 
    \[ \bigwedge_{i\in \{1, \ldots, n\} } \left(\bigvee_{j\in \{1, \ldots, m_{i}\}} \ell_{k_{ij}}\right) .\]

    Формула находится в дизъюнктивной нормальной форме (ДНФ) если она имеет вид
    \[ \bigvee_{i\in \{1, \ldots, n\} } \left(\bigwedge_{j\in \{1, \ldots, m_{i}\}} \ell_{k_{ij}}\right) .\]

    Дизъюнктом называется формула вида
    \[ \bigvee_{j\in \{1, \ldots, m\} } \ell_{j} .\] 

    Конъюнктом:
    \[ \bigwedge_{j\in \{1, \ldots, m\} } \ell_{j} .\] 

    где $\ell_{i}$ называется литералом, и имеет вид либо $x_{i}$ либо $\neg x_{i}$.
        
\end{definition}

\begin{example} \thmslashn

    Пример КНФ: $(x_1 \lor x_2 \lor \neg x_5) \land (x_3 \lor x_4 \lor x_5)$

    Пример ДНФ: $(x_1 \land x_4) \lor (x_2 \land x_4) \lor \neg x_3$
\end{example}

\begin{theorem} \thmslashn

    Любую булеву функцию можно записать в ДНФ.

    \begin{proof} \thmslashn
    
        Возьмём все наборы переменных на которых функция принимает значение $1$.

        Каждому такому набору сопоставим конъюнкт в который входят все переменные, причём, если во входном наборе переменная имеет значение $1$, то она входит как $x_{i}$, если имеет значение $0$, то как $\neg x_{i}$.

        Очевидно, что каждый конъюнкт примет значение $1$ только на одном входе, и функция примет значение $1$ если хотя-бы один конъюнкт принял значение $1$.

    \end{proof}
\end{theorem}

\begin{theorem} \thmslashn

    Любую булеву функцию можно записать в КНФ

    \begin{proof} \thmslashn
    
        Возьмём все наборы переменных на которых функция принимает значение $0$.

        Каждому такому набору сопоставим дизъюнкт в который переменная $x_{i}$ входит как литерал $x_{i}$ если $x_{i}$ имеет значение $0$ и $\neg x_{i}$ когда $x_{i}$ имеет значение $1$.

        Заметим, что такой дизъюнкт выполняется только тогда, кодга строка не совпадает с той, которой он соответствует.

        Значит, все дизъюнкты будут выполнены тогда и только тогда, когда функция не принимает значение $0$, тоесть, принимает значение $1$.
    \end{proof}
\end{theorem}
\Subsection{Билет 20: <<Полиномы Жегалкина>>}

\begin{definition} \thmslashn 

    Моном - формула вида
    \[ 1 \land \left( \bigwedge_{i\in I} x_{i}\right)  .\]
\end{definition}

\begin{example} \thmslashn

    Примеры мономов: $1$, $x_1$, $x_1x_3$.
\end{example}

\begin{definition} \thmslashn 

    Полином Жегалкина - XOR (сумма в $\mathbb{Z}/2$) мономов.

    \begin{equation*}
        \begin{array}{|c|c|c|} \hline
            a & b & a \oplus b\\ \hline
            1 & 1 & 0\\ \hline
            1 & 0 & 1 \\ \hline
            0 & 1 & 1 \\ \hline
            0 & 0 & 0 \\ \hline
        \end{array}
    \end{equation*}
\end{definition}
\begin{example} \thmslashn

    Примеры полиномов Жегалкина:

    \[ 1 \oplus x_1 \oplus x_1x_2 .\] 

    \[ x_1 \oplus x_2 \oplus x_1x_2 .\] 
\end{example}

\begin{theorem} \thmslashn

    Любую булеву функцию можно однозначно записать полиномом жегалкина.

    \begin{proof} \thmslashn
    
        Докажем существование подходящего полинома:

        Выразим основные связки:
        \begin{equation*}
            \begin{split} 
                \neg x &= 1 \oplus x\\
                x_1 \land x_2 &= x_1 \land x_2 \text{ (моном)}\\
                x_1 \lor x_2 &= x_1 \oplus x_2 \oplus x_1x_2\\
            \end{split}
        \end{equation*}

        Теперь, запишем формулу в ДНФ, раскоре $\lor$, уберём повторяющиеся члены (ессли один член встречается в мономе дважды, то второе вхождение ни на что не влияет и надо оставить одно, если один моном встречается дважды, то он отменяет себя в сложении по модулю $2$, и надо убрать оба вхождения).

        Докажем единственность:

        Всего существует $|\mathbb{B}|^{|\mathbb{B}^{n}|} = 2^{2^{n}}$ булевых функций от $n$ переменных.

        Заметим, что существует $2^{n}$ различных мономов - каждый моном либо включает либо не включает одну из $n$ переменных.

        Значит, всего существует $2^{2^{n}}$ различных многочленов Жегалкина от $n$ переменных. По принципу Дирихле, каждой функции соответствует ровно один многочлен, так-как существование хотя-бы одного уже доказано.


    \end{proof}
\end{theorem}

\Subsection{Билет 21: <<Критерий Поста>>}

\begin{definition} \thmslashn 

    Булева функция $f$ называется сохраняющей $0$, если
    \[ f(0, \ldots, 0) = 0 .\]

    Обозначим множество таких функций $T_0$.
\end{definition}
\begin{definition} \thmslashn 

    Булева функция $f$ называется сохраняющей $1$, если
    \[ f(1, \ldots, 1) = 1 .\]

    Обозначим множество таких функций $T_1$.
\end{definition}
\begin{definition} \thmslashn 

    Булева функция $f$ называется самодвойственной, если
    \[ f(\neg x_1, \neg x_2, \ldots, \neg x_n) = \neg f(x_1, x_2, \ldots, x_{n}) .\]

    Обозначим множество таких функций $S$.
\end{definition}
\begin{definition} \thmslashn 

    Булева функция $f$ называется монотонной, если
    \[ f(x_1, \ldots, 0, \ldots, x_{n}) = 1 \implies f(x_1, \ldots, 1, \ldots, x_{n}) .\]

    (замена $0$ на $1$ не может изменить результат с $1$ на $0$).

    Обозначим множество таких функций $M$.
\end{definition}
\begin{definition} \thmslashn 

    Булева функция $f$ называется линейной, если в её полиноме Жегалкина все мономы имеют не более одной переменной.

    Обозначим множество таких функций $L$.
\end{definition}
\begin{definition} \thmslashn 

    Система связок называется полной, если с её помощью можно выразить любую функцию.
\end{definition}
\begin{lemma} \thmslashn

    $\{\neg, \land, \lor\} $ - полная система связок.
    \begin{proof} \thmslashn
    
        Можно построить ДНФ.
    \end{proof}
\end{lemma}
\begin{theorem}[Критерий Поста] \thmslashn

    Система связок $B$ полная тогда и только тогда, когда в ней для каждого из вышеперечисленных классов есть хотя-бы одна функция не входящая в него:
    
    \begin{equation*}
        \begin{split}
            \exists{f\in B}\quad& f \not\in T_0\\
            \exists{g\in B}\quad& h \not\in T_1\\
            \exists{h\in B}\quad& h \not\in S\\
            \exists{k\in B}\quad& k \not\in M\\
            \exists{r\in B}\quad& r \not\in L
        \end{split}
    \end{equation*}
    \begin{proof} \thmslashn
    
        Рассмотрим функцию $f$. Если $f\in T_1$, то $f(x, \ldots, x) = 1$, иначе $f(x, \ldots, x) = \neg x$.

        Рассмотрим функцию $g$. Если $g\in T_0$, то $g(x, \ldots, x) = 0$, инчае $g(x, \ldots, x) = \neg x$.

        Мы либо получили $\neg$, либо получили $\{1, 0\}$.

        Получим $0$ или $1$ из $\neg$:

        Возьмём функцию $h$. Существует такой набор входов $\eps_{i}$, что
        \[ h(\eps_1, \ldots, \eps_{n}) = h(\neg \eps_1, \ldots, \neg \eps_{n}) .\]

        Тогда $h(\eps_1(x), \ldots, \eps_{n}(x) = h(\eps_1(\neg x), \ldots, \eps_{n}(\neg x))$, где $\eps_{i}(x)$ - $x$ если $\eps_{i} = 1$, иначе $\neg x$.

        Тогда такая функция будет константной, другую константу можно получить применив $\neg$.

        Получим $\neg$ из $\{0, 1\} $:

        Возьмём функцию $k$. Существует такой набор уходов $\eps_{i}$, что
        \[ k(\eps_1, \ldots, \eps_{i-1}, 0, \eps_{i+1}, \ldots, \eps_{n}) = 1 .\] 
        \[ k(\eps_1, \ldots, \eps_{i-1}, 1, \eps_{i+1}, \ldots, \eps_{n}) = 0 .\]

        Тогда $k(\eps_1, \ldots, \eps_{i-1}, x, \eps_{i+1}, \eps_{n}) = \neg x$.

        Теперь точно есть $\{\neg, 0, 1\} $.

        У функции $r$ есть хотя-бы один член состоящий из конъюнкции хотя-бы двух переменных. Пусть, без ограничения общности, в нём присутствуют переменные $x_1$, и $x_2$. Тогда

        \[ r(x_1, x_2, 1, \ldots, 1) = x_1x_2[\oplus x_1][\oplus x_2][\oplus 1] .\]

        Члены в квадратных скобках могут присутствтовать или отсутствовать в зависимости от формулы.

        Если присутствует член $\oplus 1$, его можно убрать применив к результату $\neg$.

        \[ x_1x_2 = x_1 \land x_2 .\]
        \[ x_1x_2\oplus x_1 = x_1 \land \neg x_2 .\] 
        \[ x_1x_2\oplus x_2 = \neg x_1 \land x_2 .\] 
        \[ x_1x_2\oplus x_1\oplus x_2 = x_1 \lor x_2 = \neg \left( \neg x_1 \land \neg x_2 \right)  .\]

        Заметим, что применяя $\neg$ можно из любого варианта получить $x_1 \land x_2$. Значит, мы выразили $\{\neg, \land\}$, из этого можно по правилам Де-Моргана выразить $\lor$, значит мы получили полную систему связок. 
    \end{proof}
\end{theorem}

\Subsection{Билет 22 <<Определение схемы. Размер, глубина схемы>>}
\begin{definition} \thmslashn
	
	Схема из функциональных элементов в базисе $B$  с $n$ входами размера $m$ - набор, состоящий из 
	$n$ булевых переменных-\textit{входов} и 
	$m$ булевых переменных-\textit{проводников}. 
	При этом для каждого проводника задана функция из $B$, 
	которая выражает его значение через другие переменные. 
	Циклы запрещены. 
	Один из проводников нарекаем \textit{выходом}.
\end{definition}
\begin{definition} \thmslashn

	Глубина схемы - максимальное количество элементов на пути от входа к выходу.
\end{definition}
\Subsection{Билет 23 <<Теорема о размере схемы в разных базисах>>}
\begin{definition} \thmslashn

	Базис полный, если любая булева функция может быть задана схемой, состоящей из B-элементов.
\end{definition}

\begin{definition} \thmslashn

	\textit{Сложность} булевой функции - минимальный размер схемы, состоящей из B-элементов, которая задаёт эту функцию.
\end{definition}

\begin{theorem}

	$B_1, B_2$ - два полных базиса. Тогда $\exists C \in \R : size_{B_1}(f)\cdot C^{-1} \leq size_{B_2}(f) \leq size_{B_1}(f)\cdot C$
	\begin{proof} \thmslashn
	
		Так как оба базиса полные, можем выразить функции одного базиса через другой. Тогда $C$ - максимальный размер такой схемы.
	\end{proof}
\end{theorem}
\Subsection{Билет 24 <<Теорема о наличии функций с большой схемной сложностью>>}

\begin{theorem} \thmslashn

	Пусть $c > 2$. Тогда сложность любой булевой функции $n$ аргументов $\leq c^n$ для всех достаточно больших $n$
	\begin{proof} \thmslashn
	
		*кукарек* Извините
		
		Размер схемы, реализующей ДНФ, с $n$ переменными есть $\mathcal{O}(n2^n)$, поскольку имеется $\leq 2^n$ конъюктов размера $\mathcal{O}(n)$
		
		Заметим что $\mathcal{O}(n2^n) = \mathcal{O}(c^n)$, потому что $c > 2$.
	\end{proof}
\end{theorem}

\begin{theorem} \thmslashn

	Пусть $c < 2$. Тогда сложность большинства булевых функций $n$ аргументов $\geq c^n$ для всех достаточно больших $n$
	\begin{proof} \thmslashn
	
		\textit{Замечание: выбор базиса изменяет размер не более, чем в константу раз, поэтому можно рассматривать базис $\{\land, \lor, \neg\}$}
		
		Оценим число различных схем размера $N$ c $n$ аргументами. *сейчас будет птичья ферма* **много кукареков** 
		
		Каждая такая схема может быть описана последовательностью из $N$ присваиваний, 
		выражающих одну из переменных через предыдущие. 
		Для каждого присваивания есть не более $3(N + n)^2$ вариантов
		(три типа операций — конъюнкция, дизъюнкция, отрицание, 
		и каждый из не более чем двух аргументов выбирается среди не более чем $N + n$ вариантов). 
		Отсюда легко получить оценку $2^{\mathcal{O}(N\log N)}$ на число всех функций сложности не более $N$ (считая $N > n$).
		Всего булевых функций с $n$ аргументами имеется $2^{2^n}$. 
		Из сравнения этих формул видно, что что при $c < 2$ и при достаточно больших $n$ булевы функции сложности меньше $c^n$ 
		составляют меньшинство, так как $2^{\mathcal{O}(c^n\log c^n)}$) много меньше $2^{2^n}$
	\end{proof}
\end{theorem}
\Subsection{Билет 25 <<Схема для сравнения чисел>>}
\textbf{*WARNING!!! Где-то тут ушёл Игорь, а я могу ооооочень сильно ошибаться. Видите ошибки - пишите мне, пожалуйста*}

Схема рекурсивная (отдельно сравниваем левые и правые половины, затем из этого получаем результат).

Будет $2n$ входов и 2 выхода. 

Делим числа на 2 половины. 
Результат функции определяют старшие разряды, если же они равны, смотрим на младшие.

Тогда мы можем 4 бита входа (результаты сравнения половин чисел) и 2 бита выхода реализовать схемой фиксированного размера.

Тогда $T(2n) \leq 2T(n) + c$. 

Тогда $T(2^k) \leq c'2^k$

Если размер чисел не степень 2 - добьём нулями.
\Subsection{Билет 26 <<Схема размера $\mathcal{O}(n)$ для сложения чисел>>}
Для вычисления результата в каждом разряде нам нужна схема константного размера (3 бита - перенос и 2 числа - значит, фиксированное количество случаев перебрать).
Тогда наша схема будет идти по числам слева направо и вычислять.
\Subsection{Билет 27 <<Схема размера $\mathcal{O}(n)$ и $глубины \mathcal{O}(\log n)$ для сложения чисел>>}
Вычисление битов переноса равносильно сравнению, поэтому нам достаточно научиться параллельно сравнивать все суффиксы чисел.

Результаты сравнения отрезков, длины которых равны степеням 2, нам уже известны из схемы сравнения.
Комбинируя их (кусок длины 2 + кусок длины 4 = кусок длины 6), получаем остальные длины.

В общем случае картина такая: после <<сужающегося дерева>> мы строим <<расширяющееся>>; 
за $k$ шагов до конца мы знаем результаты сравнения всех суффиксов, длины которых кратны $2^k$. 
Это дерево имеет размер $\mathcal{O}(n)$ и глубину $\mathcal{O}(\log n)$
\Subsection{Билет 28 <<Схема для функции голосования>>}
Эта схема имеет нечётное число аргументов и выдаёт тот, которого на входах больше.

На самом деле можно даже вычислить общее число единиц среди входов. 
Это делается рекурсивно: считаем отдельно для каждой половины, потом складываем. 
Получается логарифмическое число уровней. 
На верхнем уровне надо складывать числа размера $\log n$, на следующем — размера $(\log n -1)$ и так до самого низа, где складываются однобитовые числа (то есть биты входа). 
Какой средний размер складываемых чисел? 
Половина вершин в дереве приходится на нижний уровень (числа длины 1), четверть — на следующий (числа длины 2) и т. д. 
Вспоминая, что ряд $\sum \frac{k}{2^k}$ сходится, видим, что средний размер складываемых чисел есть $\mathcal{O}(1)$ и общий размер схемы есть $\mathcal{O}(n)$. 
А общая глубина есть $\mathcal{O}(\log n \log\log n)$, так как на каждом из $\log n$ уровней стоит схема глубины $\mathcal{O}(\log\log n)$.
