\Section{Булева логика}{Игорь Энгель}
\Subsection{Билет 18: <<Высказывания и операции. Тавтологии.>>}
\begin{definition} \thmslashn 

    Определим множество <<пропозициональных формул>> (высказывинй) следующим обарзом:

    \begin{itemize}
        \item <<пропозициональная переменная>> является высказыванием
        \item Если $A$ - высказывание, то $\neg A$ (НЕ $A$) - высказывание.
        \item Если $A$ и $B$ - высказывания, то $A \land B$ ($A$ И $B$), $A \lor B$ ($A$ ИЛИ $B$), $A \to B$ (из $A$ следует $B$) - высказывания.
    \end{itemize}
\end{definition}
\begin{definition} \thmslashn 

    Пусть высказывание $A$ содержит пропозиональные переменные $x_1, \ldots, x_{n}$.

    Соответствующий высказыванию булевой функцией называется функция $\phi_{A} : \mathbb{B}^{n} \mapsto \mathbb{B}$, где $\mathbb{B} = \{0, 1\} = \mathbb{Z}/2$, заданная индуктивно следуюзим образом:

    \begin{equation*}
        \begin{array}{|c|c|c|c|c|c|} \hline
            A & B & A \land B & A \lor B & A \to B & \neg A \\ \hline
            1 & 1 & 1 & 1 & 0 & 0\\ \hline
            1 & 0 & 0 & 1 & 0 & 0 \\ \hline
            0 & 1 & 0 & 1 & 1 & 1 \\ \hline
            0 & 0 & 0 & 0 & 1 & 1\\ \hline
        \end{array}
    \end{equation*}


\end{definition}

\begin{definition} \thmslashn 
    
    Тавтологией называется высказывание, соответветсвующия которому формула принимает значение $1$ на всех возможных входах. 

\end{definition}

\begin{example}[Примеры тавтологий] \thmslashn

    \TODO{Надо.}
\end{example}

\Subsection{Билет 19: <<Выразимость любой формулы в КНФ и ДНФ>>}

\begin{definition} \thmslashn 

    Формула находится в конъюнктивной нормальной форме (КНФ) если она имеет вид 
    \[ \bigwedge_{i\in \{1, \ldots, n\} } \left(\bigvee_{j\in \{1, \ldots, m_{i}\}} \ell_{k_{ij}}\right) .\]

    Формула находится в дизъюнктивной нормальной форме (ДНФ) если она имеет вид
    \[ \bigvee_{i\in \{1, \ldots, n\} } \left(\bigwedge_{j\in \{1, \ldots, m_{i}\}} \ell_{k_{ij}}\right) .\]

    Дизъюнктом называется формула вида
    \[ \bigvee_{j\in \{1, \ldots, m\} } \ell_{j} .\] 

    Конъюнктом:
    \[ \bigwedge_{j\in \{1, \ldots, m\} } \ell_{j} .\] 

    где $\ell_{i}$ называется литералом, и имеет вид либо $x_{i}$ либо $\neg x_{i}$.
        
\end{definition}

\begin{example} \thmslashn

    Пример КНФ: $(x_1 \lor x_2 \lor \neg x_5) \land (x_3 \lor x_4 \lor x_5)$

    Пример ДНФ: $(x_1 \land x_4) \lor (x_2 \land x_4) \lor \neg x_3$
\end{example}

\begin{theorem} \thmslashn

    Любую булеву функцию можно записать в ДНФ.

    \begin{proof} \thmslashn
    
        Возьмём все наборы переменных на которых функция принимает значение $1$.

        Каждому такому набору сопоставим конъюнкт в который входят все переменные, причём, если во входном наборе переменная имеет значение $1$, то она входит как $x_{i}$, если имеет значение $0$, то как $\neg x_{i}$.

        Очевидно, что каждый конъюнкт примет значение $1$ только на одном входе, и функция примет значение $1$ если хотя-бы один конъюнкт принял значение $1$.

    \end{proof}
\end{theorem}

\begin{theorem} \thmslashn

    Любую булеву функцию можно записать в КНФ

    \begin{proof} \thmslashn
    
        Возьмём все наборы переменных на которых функция принимает значение $0$.

        Каждому такому набору сопоставим дизъюнкт в который переменная $x_{i}$ входит как литерал $x_{i}$ если $x_{i}$ имеет значение $0$ и $\neg x_{i}$ когда $x_{i}$ имеет значение $1$.

        Заметим, что такой дизъюнкт выполняется только тогда, кодга строка не совпадает с той, которой он соответствует.

        Значит, все дизъюнкты будут выполнены тогда и только тогда, когда функция не принимает значение $0$, тоесть, принимает значение $1$.
    \end{proof}
\end{theorem}

