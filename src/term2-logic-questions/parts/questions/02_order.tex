\Section{Частично упорядоченные множества}{Чистякова Полина}

\Subsection{Билет 12 <<Отношение порядка>>}

\begin{definition} \thmslashn

	Бинарное отношение на множестве $X$ - подмножетсво $R \subset X \times X$
\end{definition}

\begin{definition} \thmslashn

	\textit{Отношение частичного порядка} на множестве $X$ - бинарное отношение на множестве $X$, обладающее следующими свойствами:
	\begin{itemize}
		\item
			\textit{рефлексивность}
		\item
			\textit{антисимметричность}
		\item
			\textit{транзитивность}
	\end{itemize}
	
	Множество $X$ тогда называется \textit{частично упорядоченным}.
	
\end{definition}

\begin{definition} \thmslashn

	Два элемента $x, y \in X$ называются \textit{сравнимыми}, если $x \geq y$ или $y \geq x$.
\end{definition}

\begin{definition} \thmslashn

	Если сравнимы любые 2 элемента из множества $X$ сравнимы, то такое отношение частичного порядка на $X$ называют \textit{линейным}.
\end{definition}

\begin{definition} \thmslashn

	Отношение строгого порядка - $x \textgreater y \iff x \geq y, x \neq y$
\end{definition}

\TODO{\textit{минимальный} и \textit{максимальный элементы} сюда писать?..}
\Subsection{Билет 13 <<Примеры упорядоченных множеств>>}

\begin{itemize}
	\item
		Числовые множества с приличным порядком (он ещё и линейный, равняйтесь на него!)
	\item
		Пример нелинейного порядка - на множестве $\R \times \R$ $<x_1, y_1> \geq <x_2, y_2> \iff x_1 \geq x_2, y_1 \geq y_2$
	\item
		На множестве функций с действительными аргументами и значениями можно ввести частичный порядок, считая, что $f \geq g$,
		 если $f(x) \geq g(x)$ при всех $x \in \R$. Этот порядок не будет линейным.
	\item
		На множестве целых положительных чисел можно определить порядок, считая, что $x \geq y$, если $y$ делит $x$. Этот порядок тоже не будет линейным.
	\item
		Пусть $U$ — произвольное множество. Тогда на множестве $P(U)$ всех подмножеств множества $U$ отношение включения $\subset$ будет частичным порядком.
	\item
		На буквах русского алфавита традиция определяет некоторый порядок (а < б < ... < ч). Этот порядок линеен — про любые две буквы можно сказать, какая из них раньше (при необходимости заглянув в словарь).
	\item
		На словах русского алфавита определён лексикографический порядок (как в словаре). Формально определить его можно так: если слово $x$ является началом слова $y$, то $x \textless y$ (например, кант \textless кантор). Если ни одно из слов не является началом другого, посмотрим на первую по порядку букву, в которой слова отличаются: то слово, где эта буква меньше в алфавитном порядке, и будет меньше. Этот порядок также линеен.
	\item
		Отношение равенства также является отношением частичного порядка, для которого никакие два различных элемента не сравнимы.
\end{itemize}

\Subsection{Билет 14 <<Операции над частично упорядоченными множествами>>}

\begin{itemize}
	\item
		Индуцированный порядок: $(\geq_Y) = (\geq) \cap (Y \times Y)$, где ($\geq$) - частичный порядок на $X, Y \subset X$
	\item
		Пусть $X$ и $Y$ — два непересекающихся частично упорядоченных множества. Тогда на их объединении можно определить 
		частичный порядок так: внутри каждого множества элементы сравниваются как раньше, а любой элемент множества $X$ по 
		определению меньше любого элемента $Y$. Это множество естественно обозначить $X + Y$.
		(Порядок будет линейным, если он был таковым на каждом из множеств.)
		
		*такое же обозначение есть и для пересекающихся множеств - просто создаём отличающиеся копии элементов, лежащих в обоих множествах*
	\item
		Пусть есть $(X, \geq_X), (Y, \geq_Y)$ - два частично упорядоченных множества.
		
		На произведении $X \times Y$ бывает 2 порядка - покоординатный и лексикографический. \TODO{на экзамене стоит это расписать}
\end{itemize}

\Subsection{Билет 15 <<Изоморфизм частично упорядоченных множеств>>}

\begin{definition} \thmslashn

	\textit{Изоморфизм} - взаимнооднозначное соответствие, сохраняющее порядок.
	
	Два множества, между которыми существует изоморфизм называются \textit{изоморфными}
\end{definition}

\begin{lemma} \thmslashn

	Отношение <<изоморфность>> - отношение эквивалентности. Классы эквивалентности называются \textit{порядковыми типами}.
	\begin{proof} \thmslashn
		
		Это отношение симметрично (мн-во изоморфно само себе), симметрично (потому что обратная функция биекции - биекция) и транзитивно (каждому элементу соответсвует свой путь до третьего множества)
	\end{proof}
\end{lemma}

\begin{theorem} \thmslashn

	Конечные линейно упорядоченные множества из одинакового числа элементов изоморфны.
	\begin{proof} \thmslashn
	
		Всегда можно взять наименьший. Тогда 
	\end{proof}
\end{theorem}

\Subsection{Билет 13 <<>>}
