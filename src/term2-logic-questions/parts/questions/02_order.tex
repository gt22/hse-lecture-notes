\Section{Частично упорядоченные множества}{Автор}
\Subsection{Билет 17 "Антицепи. Цепи. Дилуорс"}
Антицепь - подмножество элементов, в котором каждый элемент не сравним с каждым. \\
Цепь  - подмножество с линейным порядком. \\
P.S. Нарисуйте себе граф множества всех подмножеств множества из 3 элементов и поймите, что антицепью в нем будет самый широкий ряд, а цепь спустите с верхней точки до нижней.

\begin{theorem}{Теорема Дилуорса}\\ \thmslashn
Для конечного ЧУМ размер наибольешей антицепи в Х (наш ЧУМ) равен наименьшему количеству цепей, покрывающих Х
	\begin{proof} \thmslashn
\\-Вы готовы, дети???\\
- Да, капитан!\\
- Я не слышу!\\
- Так точно, капитан!!!!\\
В сторону $\leq$ все тривиально: покрыли порядок $n$ непересекающимися цепями. Тогда антицепь пересекается с каждой цепью не более, чем по 1-му элементу (иначе нарушилось бы условие на антицепь), ну тогда в этой антицепи в принципе не больше $n$ элементов. \\
То были цветочки, теперь -  ягодки: $\geq$ индукция по количеству элементов в Х.\\
База: Х из 1-го элемента выполнена - есть и цепь из 1-го элемента и антицепь из 1-го элемента.\\

Переход: Рассматриваем ЧУМ из $n+1$ элемента. Найдем в нем минимальный элемент. Почему он вообще есть? Ну вот взяли произвольный элемент $x_1$, если он не минимальный, возьмем $x_2 \leq x_1$ и т.д.Когда-нибудь мы закончим, так как по условию теоремы у нас конечное множество. Обзовем минимальный элемент $m$. Мы предполагаем, что в $X \setminus m$ найдется антицепь размер $s$. Тогда $X$ можно покрыть $s+1$ или $s$ цепями. В первом случае мы победим сразу, так как накинем еще одну цепь на одиноко стоящий $m$. Однако все не так просто, ибо нам надо рассмотреть случай когда мы покрываем $s$ цепями наш $X$. (В этой части полезно рисовать себе тентакли и отмечать на них $x_i$ о которых чуть позже пойдет речь).\\

Элемент $m$ сравним с элементами из $X\setminus m$, а к тому же его минимальность дает нам, что он кого-то меньше (действительно...). Рисуем себе тентакли и внимательно на них смотрим: на каждом тентакле (=цепь) из $X \setminus m$ у нас найдется минимальный элемент, входящий в антицепь максимального размера (вдумываемся в эту фразу). Тогда все такие элементики ($x_i$) дадут нам некоторую антицепь $A$. Почему это антицепь? Тогда для $x_i \leq x_j$ мы могли бы рассмотреть максимальную антицепь, в которой живет $x_i$, эта антицепь пересечет ту антицепь, в которой живет $x_j$, в свою очередь $x_j$ минимален у себя в домике, значит мы получим отношение сравнимости элементов антицепи ( по транзитивности), а это нарушает условие на антицепь. Ура, $A$ действительно антицепь.

Теперь выберем элемент $t \in A$ сравнимый с $m$. Такой найдется, потому что иначе в $X$ нашлась бы антицепь размера $s+1$, а этот случай мы уже благополучно рассмотрели. Тогда получим, что $m \leq t$. Все элементы цепи, содержащей $t$, такие что они меньше $t$ в антицепь максимального размера не входят. Сейчас мы выделим обрубок, который был у Близнеца на лекции: берем все элементы цепи, содержащей $t$ такие, что они выше $t$ ($t$ тоже берем). Порядок на остальных элементах не содержит антицепей размера $s$, иначе бы любая из таких антицепей пересекала обрубок ниже $t$ и мы бы опять получили нарушение условия на антицепь. В таком порядке осталось не более $n$ элементов, тогда по индукционному переходу для него выполнено утверждение теоремы и мы бьем его на $s-1$ цепь. Докидываем сюда наш обрубок и получаем разбиение на $s$ цепей. Ура.
	\end{proof}
