\Section{Частично упорядоченные множества}{Чистякова Полина, Игорь Энгель}

\Subsection{Билет 12 <<Отношение порядка>>}

\begin{definition} \thmslashn

	Бинарное отношение на множестве $X$ - подмножетсво $R \subset X \times X$
\end{definition}

\begin{definition} \thmslashn

	\textit{Отношение частичного порядка} на множестве $X$ - бинарное отношение на множестве $X$, обладающее следующими свойствами:
	\begin{itemize}
		\item
			\textit{рефлексивность}
		\item
			\textit{антисимметричность}
		\item
			\textit{транзитивность}
	\end{itemize}
	
	Множество $X$ тогда называется \textit{частично упорядоченным}.
	
\end{definition}

\begin{definition} \thmslashn

	Два элемента $x, y \in X$ называются \textit{сравнимыми}, если $x \geq y$ или $y \geq x$.
\end{definition}

\begin{definition} \thmslashn

	Если сравнимы любые 2 элемента из множества $X$ сравнимы, то такое отношение частичного порядка на $X$ называют \textit{линейным}.
\end{definition}

\begin{definition} \thmslashn

	Отношение строгого порядка - $x \textgreater y \iff x \geq y, x \neq y$
\end{definition}

\TODO{\textit{минимальный} и \textit{максимальный элементы} сюда писать?..}
\Subsection{Билет 13 <<Примеры упорядоченных множеств>>}

\begin{itemize}
	\item
		Числовые множества с приличным порядком (он ещё и линейный, равняйтесь на него!)
	\item
		Пример нелинейного порядка - на множестве $\R \times \R$: $\left<x_1, y_1\right> \geq \left<x_2, y_2\right> \iff x_1 \geq x_2, y_1 \geq y_2$
	\item
		На множестве функций с действительными аргументами и значениями можно ввести частичный порядок, считая, что $f \geq g$,
		 если $f(x) \geq g(x)$ при всех $x \in \R$. Этот порядок не будет линейным.
	\item
		На множестве целых положительных чисел можно определить порядок, считая, что $x \geq y$, если $y$ делит $x$. Этот порядок тоже не будет линейным.
	\item
		Пусть $U$ — произвольное множество. Тогда на множестве $P(U)$ всех подмножеств множества $U$ отношение включения $\subset$ будет частичным порядком.
	\item
		На буквах русского алфавита традиция определяет некоторый порядок (а < б < ... < я). Этот порядок линеен — про любые две буквы можно сказать, какая из них раньше (при необходимости заглянув в словарь).
	\item
		На словах русского алфавита определён лексикографический порядок (как в словаре). Формально определить его можно так: если слово $x$ является началом слова $y$, то $x \textless y$ (например, кант \textless кантор). Если ни одно из слов не является началом другого, посмотрим на первую по порядку букву, в которой слова отличаются: то слово, где эта буква меньше в алфавитном порядке, и будет меньше. Этот порядок также линеен.
	\item
		Отношение равенства также является отношением частичного порядка, для которого никакие два различных элемента не сравнимы.
\end{itemize}

\Subsection{Билет 14 <<Операции над частично упорядоченными множествами>>}

\begin{itemize}
	\item
		Индуцированный порядок: $(\geq_Y) = (\geq) \cap (Y \times Y)$, где ($\geq$) - частичный порядок на $X, Y \subset X$
	\item
		Пусть $X$ и $Y$ — два непересекающихся частично упорядоченных множества. Тогда на их объединении можно определить 
		частичный порядок так: внутри каждого множества элементы сравниваются как раньше, а любой элемент множества $X$ по 
		определению меньше любого элемента $Y$. Это множество естественно обозначить $X + Y$.
		(Порядок будет линейным, если он был таковым на каждом из множеств.)
		
		*такое же обозначение есть и для пересекающихся множеств - просто создаём отличающиеся копии элементов, лежащих в обоих множествах*
	\item
		Пусть есть $(X, \geq_X), (Y, \geq_Y)$ - два частично упорядоченных множества.
		
		На произведении $X \times Y$ бывает 2 порядка - покоординатный и лексикографический. \TODO{на экзамене стоит это расписать}
\end{itemize}

\Subsection{Билет 15 <<Изоморфизм частично упорядоченных множеств>>}

\begin{definition} \thmslashn

	\textit{Изоморфизм} - взаимнооднозначное соответствие, сохраняющее порядок.
	
	Два множества, между которыми существует изоморфизм называются \textit{изоморфными}
\end{definition}

\begin{lemma} \thmslashn

	Отношение <<изоморфность>> - отношение эквивалентности. Классы эквивалентности называются \textit{порядковыми типами}.
	\begin{proof} \thmslashn
		
		Это отношение рефлексивно (мн-во изоморфно само себе), симметрично (потому что обратная функция биекции - биекция) и транзитивно (каждому элементу соответсвует свой путь до третьего множества)
	\end{proof}
\end{lemma}

\begin{theorem} \thmslashn

	Конечные линейно упорядоченные множества из одинакового числа элементов изоморфны.
	\begin{proof} \thmslashn
	
		Всегда можно взять наименьший. Тогда поочереди так можно их вытаскивать и выстроить по порядку (что даёт соответствие с номерами)
	\end{proof}
\end{theorem}

\begin{definition} \thmslashn

	\textit{Автоморфизм} - изоморфизм в себя.
\end{definition}

Несколько примеров равномощных, но не изоморфных множеств:
\begin{itemize}
	\item
		Отрезок $[0, 1]$ и $\R$ - у одного есть наибольший элемент, а у другого - нет.
	\item
		$\Z$ и $\Q$ - потому что соседние должны переходить в соседние.
\end{itemize}

\Subsection{Билет 16 <<Теорема о счётных плотных линейно упорядоченных множествах>>}

\begin{definition} \thmslashn

	\textit{Соседние} элементы - два сравнимых элемента, между которыми нет третьего.
	
	\textit{Плотное} множество - множество, в котором нет соседних (между любой парой элементов есть третий)
\end{definition}

\begin{theorem} \thmslashn

	Любые 2 счётных плотных линейных множества без наименьшего и наибольшего элемента изоморфны.
	\begin{proof} \thmslashn
		
		Пусть $X$ и $Y$ — данные нам множества.
        
        Требуемый изоморфизм между ними строится по шагам.
        
        После $n$ шагов у нас есть два $n$-элементных подмножества, элементы которых мы будем называть «охваченными», и изоморфизм между ними. 
		
        На очередном шаге мы берём какой-то неохваченный элемент одного из множеств и сравниваем его со всеми охваченными элементами его множества. 
        
        Он может оказаться либо меньше всех, либо больше, либо попасть между какими-то двумя. В каждом из случаев мы можем	найти неохваченный элемент во втором множестве, находящийся в том же положении (больше всех, между первым и вторым охваченным сверху, между  вторым и третьим охваченным сверху и т. п.). 
        
        При этом мы пользуемся тем, что у нас нет наименьшего элемента, нет наибольшего и нет соседних элементов, — в зависимости от того, какой из трёх случаев имеет место. 
        После этого мы добавляем выбранные элементы к подмножествам, считая их соответствующими друг другу.

		Чтобы в пределе получить изоморфизм между множествами $X$ и $Y$, мы должны позаботиться о том, чтобы все элементы обоих множеств были рано или поздно охвачены. Это можно сделать так:

		Поскольку каждое из множеств счётно, пронумеруем его элементы и будем выбирать неохваченный элемент с наименьшим номером (на нечётных шагах — из $X$, на чётных — из $Y$ ). 
        
        Это соображение завершает доказательство.
	\end{proof}
\end{theorem}

\begin{theorem} \thmslashn

	Всякое счётное линейно упорядоченное множество изоморфно некоторому подмножеству множества $\Q$.
	\begin{proof} \thmslashn
	
		Делаем тоже самое, только выбираем элементы не из обоих множеств, а из первого.
	\end{proof}
\end{theorem}

\Subsection{Билет 17 <<Определение цепи, антицепи. Теорема Дилуорса>>}
\begin{definition} \thmslashn 

    Пусть $\left<X, \le\right>$ - ЧУМ. Цепью называется подмножество $Y \subset X$, такое, что все элементы в $Y$ попарно сравнимы.
\end{definition}
\begin{definition} \thmslashn 

    Пусть $\left<X, \le\right>$ - ЧУМ. Цепью называется подмножество $Y \subset X$, такое, что все элементы в $Y$ попарно несравнимы.
\end{definition}

\begin{theorem}[Теорема Дилуорса] \thmslashn

   Для конечного частично упорядоченного множества $X$ размер максимальной антицепи равен минимальному количеству цепей, необходимому чтобы покрыть множество.

    \begin{proof} \thmslashn
    
        Пусть $A$ - максимальная антицепь в $X$. $|A| = d$.

        Покажем, что необходимо $d$ цепей для покрытия:

        Пусть покрыли меньше чем $d$ цепями. Тогда, по принципу Дирихле, какая-то цепь содержит хотя-бы два элемента $A$. Но $A$ антицепь, значит они несравнимы. Противоречие.

        Покажем что $X$ можно покрыть не более чем $d$ цепями:
        
        Индукция по $|X|$. Для $|X| = 0 \implies X=\emptyset$ тривиально.

        Если все элементы $X$ несравнимы, то $X = A$, и единственное все возможные цепи содержат не более одного элемента, так-что для покрытия нужно $d$ цепей.

        Пусть $m$ - минимальный элемент $X$, такой, что $\exists{a\in X}\quad m \le a$, а $M$ - максимальный элемент  $S' = \{a\in X \setminus \{m\} \ssep m \le a \} $ (это множество точно не пустое, так-как $\exists{a\in X}\quad m \le  a$. Причём, $m \le a \le M \implies m \le M$). 

        $M$ будет максимальным элементом $X$: Предположим что это не так, тоесть $\exists{a\in X}\quad M \le a$, тогда по транзитивности $m \le M \le a \implies m \le a \implies a\in S'$, но $M$ - максимальный элемент  $S'$, значит  $a \le M$ невозможно.

        Пусть $T = X \setminus \{m, M\} $. $|T| < |X|$, значит размер поркрытия $T$ не больше чем размер максимальной антицепи в $T$. Так-как $T \subset X$, максимальная антицепь в $T$ имеет размер $\le d$. Если её размер $< d$, добавим к покрытию $T$ цепь $\{m, M\} $, и теорема доказана.

        Остаётся случай, когда максимальная антицепь в $T$ имеет размер $d$.

        Определим два множества:
        \[ X^{+} = \{x\in X \ssep \exists{a\in A}\quad a \le x\} .\]
        \[ X^{-} = \{x\in X \ssep \exists{a\in A}\quad x \le a\}  .\]

        Заметим, что $X^{+} \cup X^{-} = X$, ведь если какой-то элемент не входит в ни в одно из множеств, то его можно добавить в антицепь $A$, а она максимальна.

        Так-же заметим, что $X^{+}\cap X^{-} = A$, по рефлексивнсоти и антисимметричности порядка. Так-как $X^{\pm} \subset X$, $X^{+}\cap X^{-} \subsetneq X$, то $X^{\pm} \subsetneq X$ (в $X^{+}$ должен быть элемент которого нет в $X^{-}$, значит в $X^{-}$ нет какого-то элемента из $X$, симметрично для  $X^{+}$ ) 

        Значит, $X^{+}$ и $X^{-}$ строго меньше $X$, значит для них выполняется предположение индукции. При этом, размер максимальной антицепи в них ровно $d$ ($A \subset X^{\pm} \subset X$), значит каждый из них можно покрыть $d$ цепями.

        Каждая цепь будет содержать ровно $1$ элемент из $A$ (цепь не может содержать больше одного элемента антицепи + принцип Дирихле). При этом, тот элемент будет наименьшем/наибольшим элементов цепи (по построению множеств). Значит, объединение этих цепей будет цепью в $X$, получилось покрытие $X$ $d$ цепями.
    \end{proof}
\end{theorem}
