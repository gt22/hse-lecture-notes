\Section{Равномощность}{Автор: Чистякова Полина}

Это самое начало теории множеств (какое отношение оно имеет к матлогике не очень понятно, но  пусть будет).

\Subsection{Билет 01 "Равномощные множества"}

Для множеств определено отношение равномощности 
	- <<множество $A$ равномощно множеству $B$ >> значит, что из одного множества в другое можно построить биекцию.
Иными словами это значит, что любому элементу из множества $A$ сопоставляется ровно один элемент из множества $B$.

Пример двух равномощных множеств 
	- в парке гуляют дети. Если каждому ребёнку на входе в парк подарить шарик, то множества детей и шариков в парке будут равномощны (если никакой ребёнок не отпустит/лопнет шарик).
	
Как у любого отношения у равномощности есть свои свойства.
Это отношение эквивалентности.
Это значит, что оно \begin{itemize}
		    	\item
		    		рефлексивно: <<$A$ равномощно $A$>>
		    	\item
		    		симметрично
		    	\item
		    		транзитивно 
		    \end{itemize}
		    \TODO Доказательства
\TODO Примеры *их там много и они страаашные :с*

\Subsection{Билет 02 "Счётные множества"}

Билет не просто маленький, он \textit{крошечный}... В книге просто куча воды. Кажется, на экзамене это не попадётся. \newline \newline

Счётное множество - множество, равномощное множеству натуральных чисел.
Иными словами - мы просто <<пронумеровали>> все элементы множества.
Биекция будет означать, что у каждого элемента есть номер (сюръекция), и у каждого элемента не больше одного номера (инъекция).

\TODO: надо расписать примеры, ага
Самые простые примеры счётных множеств: само \N или множество значений линейной функции от натуральных чисел. (ещё бывает \Z)

Чуть более сложный пример - \Q \newline
Но его доказательство - следующий билет с: 
\TODO ссылочка просится :с

\Subsection{Билет 0(3+4) "Счётность множества рациональных чисел" + "Счётность объединения счётного количества счётных множеств"}

Тут будет ооочень много лемм и теорем, готовьтесь...

\begin{lemma} \thmslashn

	Объединение двух счётных множеств счётно.
	
	\begin{proof} \thmslashn
	
		Рассмотрим два счетных множества A и B; каждое из них можно записать в последовательность:
		$$a_0, \quad a_1, \quad a_2, \quad ...$$
		$$b_0, \quad b_1, \quad b_2, \quad ...$$
		Теперь можно поочерёдно брать элементы из первой и второй последовательности и записывать в новую (это даст нам $A \cup B$):
		$$a_0, \quad b_0, \quad a_1, \quad b_1, \quad a_2, \quad b_2, \quad a_3, \quad b_3, \quad ...$$
		Если $A \cup B = \emptyset$, то мы всё доказали \char`\^\char`\^ \newline
		Если же это не так, то повторяющиеся элементы мы просто не выписываем.
	\end{proof}
\end{lemma}

\begin{lemma} \thmslashn

	Всякое подмножество счетного множества конечно или счетно.
	\begin{proof} \thmslashn
	
		Пусть у нас есть счётное множество $A$ и его подмножество $A'$. Выпишем множество $A$ в строчку. Затем зачеркнём все элементы, не принадлежащие $A'$. Получим последовательность из всех элементов $A'$: либо конечную (тогда $A'$ конечно), либо бесконечную (тогда $A'$ счётно)
	\end{proof}
\end{lemma}

\begin{lemma} \thmslashn

	Всякое бесконечное множество содержит счётное подмножество.
	\begin{proof} \thmslashn
	
		Выпишем бесконечную последовательность. Возьмём первый элемент случайно (множество не пусто). Дальше будем каждый раз рассматривать дополнение получившейся послдовательности до изначального множества. Оно никогда не кончится (множество бесконечно), значит, мы всегда сможем выписать новый элемент последовательности. Получили бесконечную последовательность. Значит, у изначального множества есть счётное подмножество.
	\end{proof}
\end{lemma}

\begin{lemma} \thmslashn

	Множество рациональных чисел \Q счетно
	\begin{proof} \thmslashn
	
		Докажем сначала отдельно про положительные и отрицательные части \Q. Тогда по одной из предыдущих лемм их объединение будет счётно. \newline
		Неотрицательное рациональное число задается парой чисел — числителем и знаменателем. Числитель может быть произвольным натуральным числом, а знаменатель произвольным положительным натуральным числом. Выпишем все такие числа в виде таблицы, бесконечной вниз и вправо:
		\begin{equation*}
			\begin{matrix}
				0/1 & 1/1 & 2/1 & 3/1 & \ldots \\
				0/2 & 1/2 & 2/2 & 3/2 & \ldots \\
				0/3 & 1/3 & 2/3 & 3/3 & \ldots \\
				\vdots & \vdots & \vdots & \vdots & \ddots 
			\end{matrix}
		\end{equation*}
		В этой таблице выписаны все числа (а некоторые даже повторяются......) \newline
		Числа из этой таблицы теперь уже легко выписать в последовательность. Например, можно идти по диагоналям (вниз-влево). Сначала выпишем единственное число на первой диагонали (0/1), потом два числа на второй (1/1, 0/2), потом три числа на третьей и так далее:
		$$0/1, 1/1, 0/2, 2/1, 1/2, 0/3, 3/1, 2/2, 1/3, 0/4, \ldots $$
		Другими словами, мы сначала выписываем все числа с суммой числителя и знаменателя 1, потом — с суммой 2, потом 3 и так далее. Если мы встречаем число, которое уже выписывали - просто пропускаем его. \newline
		Доказательство для отрицательной части \Q аналогично.
	\end{proof}
\end{lemma}

\begin{theorem} \thmslashn

	Объединение конечного или счётного числа конечных или счётных множеств конечно или счётно.
	\begin{proof} \thmslashn
	
		Пусть есть счётное количество счётных множеств $A_1, A_2, A_3, \ldots$. Выпишем их в табличку:
		\begin{equation*}
			\begin{matrix}
				A_0: & a_{00} & a_{01} & a_{02} & a_{03} & \ldots \\
				A_1: & a_{10} & a_{11} & a_{12} & a_{13} & \ldots \\
				A_2: & a_{20} & a_{21} & a_{22} & a_{23} & \ldots \\
				A_3: & a_{30} & a_{31} & a_{32} & a_{33} & \ldots \\
				\vdots & \vdots & \vdots & \vdots & \vdots & \ddots 
			\end{matrix}
		\end{equation*}
	\end{proof}
\end{theorem}

\begin{theorem} \thmslashn

	Декартово произведение двух счётных множеств A × B cчётно.
	\begin{proof} \thmslashn
	
		Декартово произведение - множество упорядоченных пар вида $(a, b) \ssep a \in A, b \in B$.\newline
		Разделим пары на группы - в каждой группе первый элемент пары совпадает. Тогда получим счётно объединение счётных множеств (у нас будет <<$|A|$ штук>> множеств по <<$|B|$ штук>> элементов в каждом)
	\end{proof}
\end{theorem}

\Subsection{Билет 05 "Добавление счётного множества"}

\TODO: Вася спаси.............. \newline
Upd: Сами справились (уж лучше не....) \newline

\begin{theorem} \thmslashn

	Если множество $A$ бесконечно, а множество $B$ конечно или счётно, то множество $A \cup B$ равномощно $A$.
	\begin{proof} \thmslashn
	
		НУО $A \cap B = \emptyset $ - иначе вместо $B$ берём $B \setminus A$. \newline
		Мы знаем, что в $A$ есть счётное подмножество $A_0$. Тогда есть биекция из $A_0 \cup B$ в $B$ (потому что оба множества счётные - биекция черз натуральные числа). Тогда есть биекция из $A \cup B = (A \setminus A_0) \cup (A_0 \cup B)$
	\end{proof}
\end{theorem}

\Subsection{Билет 06 "Равномощность отрезка [0,1] множеству всех бесконечных последовательностей из 0 и 1"}
Надеюсь, здесь нужно только то, что в названии билета (\TODO: написать всякие интервалы, полуинтервалы и тд)
\begin{theorem} \thmslashn

	*Вставь сюда название билета*
	\begin{proof} \thmslashn
	
		Мы знаем, что $\forall x \in [0,1]$ существует запись $x$ в виде бесконечной двоичной дроби. (\TODO: сюда бы картинку из samples/) Но тогда некоторым точкам будут соответсвовать 2 последовательности (например, 0, 1001111... и 0, 101000...). Тогда выкинем все последовательности, заканчивающиеся бесконечным рядом единиц (их счётное число, поэтому так можно)
	\end{proof}
\end{theorem}

\Subsection{Билет 07 "Равномощность квадрата отрезку"}

\begin{theorem} \thmslashn

	*Название билета*
	\begin{proof} \thmslashn
	
		Мы знаем, что каждому числу из [0, 1] соответствует одна бесконечная последовательность из 0 и 1. Тогда [0, 1] × [0, 1] соответсвует пара таких последовательностей. Биекция между парой и последовательностью:
		$$(a_0a_1a_2a_3\ldots, b_0b_1b_2b_3\ldots) \rightarrow a_0b_0a_1b_1a_2b_2a_3b_3...$$
	\end{proof}
\end{theorem}

\Subsection{Билет 08}

\Subsection{Билет 09}

\Subsection{Билет 10}

\Subsection{Билет 11}
