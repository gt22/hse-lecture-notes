\SectionLecture{Лекция 4}{Игорь Энгель}
\begin{definition} \thmslashn 

    Поле рациональных функций: $K(x) = Q(K[x])$.
\end{definition}
\begin{statement} \thmslashn

    $\frac{f}{g}\in K(x)$. $\exists!{u,v\in K[x]}$ т. ч:
    \begin{enumerate}
        \item $(u, v) = 1$
        \item  старший коэффициент $v$ - 1
        \item $\frac{f}{g} = \frac{u}{v}$
    \end{enumerate}
    \begin{proof}
        Существование:

        Пусть $d = (f,g)$. 

        Пусть $\hat{u} = \frac{f}{d}$, $\hat{v} = \frac{g}{d}$.

        Тогда $(\hat{u}, \hat{v}) = 1$ и $\frac{f}{g} = \frac{\hat{u}}{\hat{v}}$.

        Пусть $c$ - старшний коэффицент $\hat{v}$.

        Тогда $v = c^{-1}\hat{v}$, $u = c^{-1}\hat{u}$.

        Единственность:
        
        Пусть $\frac{u_1}{v_1} = \frac{u_2}{v_2}$, при этом $(u_1, v_1) = 1$ и $(u_2, v_2) = 1$.

        Тогда $v_2u_1 = u_2v_1$. Заметим, что $v_1\divby v_2$ и $v_2\divby v_1$.

        Значит, $v_1 = cv_2$. 

        Так-как старшие коэффициенты равны $1$, $c = 1$ и $v_1=v_2$, а значит и $u_1=u_2$.
    \end{proof}
\end{statement}
\begin{lemma} \thmslashn

    $\frac{r_1}{g_1}$, $\frac{r_2}{g_2}$ - правильные дробе ($\deg r < \deg g$).

    Тогда $\frac{r_1}{g_1} + \frac{r_2}{g_2}$ - правильная дробь.
    \begin{proof}
        \[\frac{r_1g_2 + r_2g_1}{g_1g_2} = \frac{r_3}{g_3}.\]
        \[ \deg r_3 \le \max(\deg r_1g_2, \deg r_2g_1) < \deg g_1g_2 = \deg g_3 .\] 
    \end{proof}
\end{lemma}
\begin{statement} \thmslashn

    $\forall{\frac{f}{g}\in K(x)}\quad \exists!{ h \in K[x], \frac{r}{g_1}\in K(x)}\quad \frac{f}{g} = h(x) + \frac{r}{g_1}, \deg r < \deg g_1$
    \begin{proof}
        Существование:
        \[ f = gh + r .\]
        \[ \frac{f}{g} = h + \frac{r}{g} .\] 
        
        Единственность:

        Пусть $h_1 + \frac{r_1}{g_1} + h_2 + \frac{r_2}{g_2}$.

        Тогда $h_1 - h_2 = \frac{r_2}{g_2} - \frac{r_1}{g_1}$. Значит, $h_1 - h_2$ - правильная дробь. Но единственная прваильная дробь из оригинального кольца многочленов - $0$.
    \end{proof}.
\end{statement}
\begin{definition}[Простейшая дробь] \thmslashn 

    Пусть $p\in K[x]$, $p$ - неприводимый.

    Простейшей дробью называется $\frac{r}{p^{\alpha}}$, где $\deg r < \deg p$, $\alpha\in \mathbb{N}$.
\end{definition}
\begin{lemma}[О разложении по основанию] \thmslashn

    Пусть $r, p\in K[x]$, $\deg p \ge 1$.

    Тогда существует разложение $r(x) = a_0(x) + a_1(x)p + \ldots + a_{k}(x)p^{k}$, где $\deg a_{i} < \deg p$.
    \begin{proof}
        Заметим, что $r = g\cdot p + a_0$.
        
        Остальные коэффициенты можно получить разложив $g$:

        \[ g = a_1 + a_2p + \ldots + a_{k}p^{k-1} .\]
        \[ r = a_0 + a_1p + \ldots + a_{k}p^{k} .\qedhere\] 
    \end{proof}
\end{lemma}
\begin{statement} \thmslashn

    Пусть $\frac{f}{p^{\alpha}}$ - правильная дробь.

    Тогда существует разложение $\frac{f}{p^{\alpha}} = \sum\limits_{j=1}^{k} \frac{r_{j}}{p^{j}}$, $\forall{j}\quad \deg r_{j} < \deg p$.
    \begin{proof}
        \[ \frac{f}{p^{\alpha}} = \frac{\sum\limits_{j=1}^{k} p^{k-j}r_{j}}{p^{k}} .\]

        Существование:

        Пусть $k=\alpha$.
        
        Тогда $r_j$ - коэффициенты из разложения $f$ по основанию $p$.
    \end{proof}
\end{statement}
\begin{lemma} \thmslashn

    $\frac{f}{g}$ - правильная дробь, $g = g_1g_2$, $(g_1, g_2) = 1$.

    Тогда существует единственное разложение $\frac{f}{g} = \frac{h_1}{g_1} + \frac{h_2}{g_2}$ (все дроби правильные).
    \begin{proof}
       \[ f = h_1g_2 + h_2g_1 .\]

       Такие уравнения решаются аналогично диофантовым в целых числах.

       Пусть нашли какое-то решение $\tilde{h_1}, \tilde{h_2}$. Тогда можем найти другое решение $h_2$ как остаток от деления $\tilde{h_2}$ на $g$. Тогда $\deg h_2 < \deg g$. 

       Найдём соответствующие $h_1$:
       \[ h_1g_2 = h_2g_1 - f .\]
       \[ \deg h_1 + \deg g_2 < \deg g_1 + \deg g_2 \implies \deg h_1 < \deg g_1 .\]

       Все отсальные решения уравнения имеют слишком большую степень.
    \end{proof}
\end{lemma}
\begin{theorem}[О разложении на простейшие] \thmslashn

    Пусть $\frac{f}{g}\in K(x)$. Тогда $\exists!{h,p_1, \ldots, p_{k}, r_{ij}\in K[x]}\quad \exists!{\alpha_1, \ldots, \alpha_{n}}\quad 1 \le i \le k, 1 \le j \le \alpha_{i}, \frac{f}{g} = h(x) + \sum \frac{r_{ij}}{p_{i}^{j}}, \deg r_{ij} < \deg p_{i}, p_{i} \text{ - неприводимые}, \text{старший коэффицент $p_i$ - $1$}, r_{i \alpha_{i}} \neq 0,  $ 
    \begin{proof}
        \[ \frac{f}{g} = h(x) + \frac{r}{g} .\]
        \[ g = p_1^{\alpha_1} \ldots p_{k}^{\alpha_{k}} .\]

        Возьмём $g_1 = p^{\alpha}$

        Предположим, что $g$ имеет старший коэффициент $1$, как и все $p_{i}$ (несложно сделать).

        Тогда
        \[ \frac{r}{g} = \frac{h_1}{p_1^{\alpha_1}} + \frac{h_2}{p_2^{\alpha_2} \ldots p_{k}^{\alpha_{k}}} .\]

        По индукции, всё разложится на дроби.

        Разложим дробь по основанию:
        \[ \frac{r}{g} = \sum\limits_{j=1}^{\alpha_1'}\frac{r_{1j}}{p_1^{j}} + \frac{h_2}{p_2^{\alpha_2}\ldots p_{k}^{\alpha_{k}}} .\] 

        Где $\alpha_1'$ - максимальное число, такое что $r_{1\alpha_1'} \neq 0$.
        
        Раскладывая по индукции, получим нужное разложение.

        Единственность $h$ следует из правильности суммы.

        Без ограничения общности, пусть $\frac{f}{g}$ - неприводимая дробь, и старший коэффицент $g$ - $1$.

        Тогда $g = p_1^{\alpha'_1}\ldots p_{k}^{\alpha'_{k}}$.

        \TODO{я запутался}
    \end{proof}
\end{theorem}
\begin{example} \thmslashn

    Простейшие дроби в $\mathbb{C}$ - $\frac{c}{(x-\lambda)^{\alpha}}$, $c,\lambda\in \mathbb{C}$

    Простейшие дроби в $\mathbb{R}$ - $\frac{c}{x-\lambda}$, $c, \lambda\in \mathbb{R}$, $\frac{Ax+B}{(x^2+ax+b)}$, $A,B,a,b\in \mathbb{R}$, $b^2 < 4a$.

    Разложим $\frac{1}{x^{n} - 1}$:

    \[ 1 = \sum\limits_{i=1}^{n} \frac{x^{n}-1}{n\eps_{i}^{n-1}(x-\eps_{i})} .\]
    \[ \frac{1}{x^{n}-1} = \sum\limits_{i=1}^{n} \frac{1}{n\eps^{n-1}(x-\eps_{i})} = \sum\limits_{i=1}^{n} \frac{\eps_{i}}{n(x-\eps_{i})} .\] 
\end{example}
