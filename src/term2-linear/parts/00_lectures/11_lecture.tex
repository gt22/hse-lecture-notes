\SectionLecture{Лекция 11}{Игорь Энгель}
\begin{properties} \thmslashn

    Если $A, B, C\in M_{n}(K)$.
\begin{enumerate}
    \item[0] $\det(A) = \det(A^{T})$ 
    \item[1] Определитель не меняется при элементарных преобразованих первого типа (лин. комбинация) для строк/столбцов, меняет знак при свопе строк/столбцов, домножается на $\lambda$ при домножении строки/столбца на $\lambda$.
        \begin{proof} \thmslashn
            
            Для столбцов прямо следует из свойств определителя, для строк рассмотрим определитель $A^{T}$.
        \end{proof}
    \item[2] $\det(AB) = \det(A)\det(B)$
        \begin{proof} \thmslashn
        
            Возьмём форму $B \to \det AB$. Она линейна по столбцам $B$. Чтобы показать это, заметим, что $A(B + X) = AB+AX \implies (B+X) \to \det A(B+X) = \det AB +\det AX$ (альтернативно: определитель линеен по столбцам произведения, произведение линейно по столбцам $B$ ($i$-й столбец произведения зависит только от $i$-го столбца $B$, и зависит линейно).

            Если у матрицы $B$ есть одинаковые столбцы, то они есть и у матрицы $AB$. 

            Значит, $B \to \det AB$ - форма объёма. Все формы объёма пропорциональны, подставим $B = E_{n}$, получим, что $B \to \det A \det B \implies \det AB = \det A \det B$.

            Альтернативно - найдём комбинацию элементарных преобразований $L$, такую, что $LA = E_{n}$ такая комбинация существует, если $A$ невырождена.

            Если $A$ вырождена, то $\det A = 0$, $AB$ - вырождена, $\det AB = 0 = 0\det B$.

            Тогда $LAB = B$.

            Тогда $L^{-1}B = AB$.

            Если $B=E_{n}$, то получаем $\det A = \det L^{-1} 1 = k$.

            По пункту $1$, применение этих элементарных преобразований всегда домножает определитель на $k$, значит $\det L^{-1}B = k\det B = \det A\det B$.
        \end{proof}
    \item[3] $\det \begin{bmatrix} A & B\\ 0 & C \end{bmatrix} = \det(A)\det(C)$
        \begin{proof} \thmslashn
        
            Сначала рассмотрим $\det \begin{bmatrix} E_{n} & B\\ 0 & E_m \end{bmatrix} $. Элементарными преобразованиями первого типа можно получить из неё $E_{n+m}$, значит определитель $1$.

            Заметим, что форма $A \to \det \begin{bmatrix} A & B\\ 0 & E_{m} \end{bmatrix} $ - форма объёма, значит она пропорциональна $\det A$, причём с коэффициентом  $1$.

            Теперь, рассмотрим форму $C \to \det \begin{bmatrix} A & B\\ 0 & C \end{bmatrix}$ - это форма объёма по строчкам $C$. Получаем, что оно пропорционально $\det C$ с коэффициентом $\det A$.

            Значит, $\det \begin{bmatrix} A & B\\ 0 & C \end{bmatrix} = \det A \det B$.
        \end{proof}
    \item[4] Определитель верхнетреугольной или нижнетреугольной матрицы равен произведению диогональных элементов.
        \begin{proof} \thmslashn
        
            Индукция по размеру.

            Пусть $X\in LT_{n}$.

            Разобьём на блоки вида $\begin{bmatrix} A & B\\ 0 & C \end{bmatrix} $. $A = X_{11}$, остальные выводятся из этого. $\det X = \det A \det C = X_{11} \det C = \prod\limits_{i=1}^{n} X_{ii} $. $C\in LT_{n-1}$.

            Если получили матрицу из $LT_1$, то определитель равен единственному её элементу.
        \end{proof}
    \item[5] $\det(A^{-1}) = (\det A)^{-1}$
        \begin{proof} \thmslashn
        
           Заметим, что $\det A^{-1}A = 1 = \det A^{-1}\det A \implies \det A^{-1} = \frac{1}{\det A}$. 
        \end{proof}
    \item[6] $\det : GL_{n}(K) \to K^{*}$ - гомоморфизм групп
        \begin{proof} \thmslashn
        
            Произведение сохраняется по свойству $2$, результат обратим по свойству $5$.
        \end{proof}
    \end{enumerate}
\end{properties}

Определитель можно вычислить методом Гаусса, привдя матрицы к ступенчатом виду, переменожив элементы на диоганали, и скорректировавшись на эффект преобразований.
\begin{theorem} \thmslashn

    Если есть отображение $\Vol : M_{n}(\mathbb{R}) \to \mathbb{R}$, удовлетворяющие свойствам
    \begin{enumerate}
        \item $\Vol(E_{n}) = 1$ 
        \item $\Vol(\ldots, u + \lambda v, \ldots, v) = \Vol(\ldots, u, \ldots, v, \ldots)$
        \item $\Vol (\ldots, \lambda v, \ldots) =  |\lambda|\Vol(\ldots, v, \ldots)$
    \end{enumerate}
    То $\Vol = |\det|$
    \begin{proof} \thmslashn
    
        Рассмотрим случай, когда матрица вырождена.

        Тогда существует элемент, который можно выразить как линейную комбинацию других, и можно получить $\Vol(\ldots, 0, \ldots) = 0 \Vol(\ldots, v, \ldots) = 0$.

        Если матрица невырождена, то можно привести её к единчному виду. Обе функции меняются одинаокого при элементарных преобразованиях, занчит они совпадут.
    \end{proof}
\end{theorem}
