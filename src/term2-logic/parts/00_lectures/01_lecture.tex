\SectionLecture{Лекция 1}{Игорь Энгель}
\begin{definition} \thmslashn 

    Множества $A$ и $B$ называются равномощными, если $\exists{\text{ биекция}}\quad f : A \to B $.
\end{definition}
\begin{remark} \thmslashn

    Равномощность - отношение эквивалентности:
    \begin{enumerate}
        \item $A \sim A \impliedby f = \id$
        \item $(A \sim B \implies B \sim A) \impliedby g = f^{-1}$.
        \item $(A \sim B \land B \sim C \implies A \sim C) \impliedby h = g \circ f$.
    \end{enumerate}
\end{remark}
\begin{definition} \thmslashn 

    Множество называется счётным, если оно равномощно $\mathbb{N}$.
\end{definition}
\begin{example} \thmslashn

    Примеры счётных множеств: $\mathbb{N}, \mathbb{Z}, \{x\in \mathbb{N}\ssep x \ge 2\} $.
\end{example}
\begin{lemma} \thmslashn

    Если $A$ и $B$ счётны, то $A \cup B$ счётно.
    \begin{proof} \thmslashn
    
        Пусть $A = \{a_1, a_2, \ldots\} $, $B = \{b_1, b_2, \ldots\} $.

        Тогда, представим $A \cup B = \{a_1, b_1, a_2, b_2, \ldots\} $.

        Тогда 
        \[ f(x) = \begin{cases}
            a_{i} & x = 2i+1\\
            b_{i} & x = 2i
        \end{cases} .\] 
        Если $A\cap B$ не пустое, то некоторые элементы функции надо выкинуть.
    \end{proof}
\end{lemma}
\begin{lemma} \thmslashn

    Если $A$ счётно, то $\forall{B \subset A}\quad $, $B$ либо конечно либо счётно.
    \begin{proof} \thmslashn
    
        Так-как $A$ счётно, возьмём биекцию $f : \mathbb{N} \mapsto A$.

        Тогда, $g(i) = f(\min \{j \ge i \ssep f(j)\in B\} )$.
    \end{proof}
\end{lemma}
\begin{lemma} \thmslashn

    Любое бесконечное множество содержит счётное подмножество.
    \begin{proof} \thmslashn
    
        Так-как $A$ бесконечно, в нём существует хотя-бы один элемент $a_1$, $A \setminus \{a_1\} $ тоже бесконечно, и можно взять $a_1$, $a_2$, $a_3$, $\ldots$. 
    \end{proof}
\end{lemma}
\begin{lemma} \thmslashn

    $\mathbb{Q}$ счётно.
    \begin{proof} \thmslashn
    
        Докажем сначала для $\mathbb{Q}_{+}$.

        Выпишем их следующим образом: \TODO
        \begin{equation*}
            \begin{bmatrix} 
                \frac{0}{1} & \frac{1}{1} & \frac{2}{1} & \ldots\\
                \frac{0}{2} & \frac{1}{2} & \frac{2}{2} & \ldots\\
                \frac{0}{3} & \frac{1}{3} & \frac{2}{3} & \ldots\\
                \vdots & \vdots & \vdots & \vdots
            \end{bmatrix} 
        \end{equation*}
        Будем обходить эту таблицу диагоналями: $\frac{0}{1}, \frac{1}{1}, \frac{0}{2}, \frac{0}{3}, \frac{1}{2}, \frac{2}{1}, \frac{3}{1}, \frac{2}{2}, \ldots$.

        Будем пропускать повторяющиеся числа.

        Каждое число будет выписано, потому-что за $(n+m)^2$ шагов мы точно дойдём до строки $m$ столбца $n$ и получим число $\frac{n}{m}$.

        Аналогичное доказательство подходит для $\mathbb{Q}_{-}$. $\mathbb{Q} = \mathbb{Q}_{+} \cup \mathbb{Q}_{-}$, значит, тоже счётно.
    \end{proof}
\end{lemma}
\begin{lemma} \thmslashn

    Объединение конечного или счётного числа счётных или конечных множеств счётно или конечно.
    \begin{proof} \thmslashn
    
        Выпишем на $i$ строке в $j$ столбце $j$-й элемент $i$-го множества. Построив такую-же последовательность как для рациональных чисел, получим биекцию, либо кончатся элементы.
    \end{proof}
\end{lemma}
\begin{theorem} \thmslashn

    Если $A, B$ - счётные, то $A \times B$ счётно.
    \begin{proof} \thmslashn
    
        Рассмотрим семество множеств  $A_{a} = \{\left( a, b \right) \ssep b\in B\}$. $\forall{a\in A}\quad A_{a} \text{ равномощно } B$, значит, $A_{a}$ счётно. При этом, различных $A_{a}$ - счётно, столько-же сколкьо элементов $A$.

        Тогда
        \[ A \times B = \bigcup_{a\in A} A_{a}  .\]
        
        счётно по предыдущей лемме.
    \end{proof}
\end{theorem}
\begin{statement} \thmslashn

    Множества $(0, 1]$ и $[1, +\infty)$ равномощны с биекцией $\frac{1}{x}$.

    Множества $[0, 1]$ и $[0, 1)$ равномощны
    \begin{proof} \thmslashn
    
        Пусть $B = [0, 1] \setminus \left\{\frac{1}{i} \ssep i \ge 1\right\} $
        \[[0, 1] = \left\{\frac{1}{i} \ssep i \ge 1\right\} \cup B.\]
        \[ [0, 1) = \left\{\frac{1}{i} \ssep i > 1\right\} \cup B  .\] 
        Возьмём биекцию:
        \begin{equation*}
            f(x) = \begin{cases}
                \frac{1}{i + 1} & x = \frac{1}{i}\\
                x
            \end{cases}
        \end{equation*}
    \end{proof}
\end{statement}
\begin{theorem} \thmslashn

    Если $A$ бесконечно, $B$ конечно или счётно, то $A \cup B$ равномощно $A$.
    \begin{proof} \thmslashn
    
        Рассмотрим случай $A\cap B = \emptyset$.

        Возьмём cчётное $A_0 \subset A$.

        Начнём строить биекцию: $\forall{x \in A \setminus A_0}\quad f(x) = x$.

        Так-как $A_0 \cup B$ счётно, то между $A_0$ и $A_0 \cup B$ существует биекция, воспользуемся ей чтобы достроить $f$.
    \end{proof}
\end{theorem}
\begin{theorem} \thmslashn

    Отрезок $[0, 1]$ равномощен множеству бесконечных последовательностей из $\{0, 1\} $.
    \begin{proof} \thmslashn
   
        Можно доказывать для $[0, 1)$, так-как они равномощны.

        Каждому $x\in [0, 1)$ соответствует представление в двоичной системе счисления. (например, $\frac{1}{4} = 0.010000000$.

        Заметим, что разные числа переходят в разные последовательности, но разные последовательности могут перейти в одно число. Для чисел вида $\frac{n}{2^{k}}$  $n, k \in \mathbb{N}$ существует две последовательности. 

        Тогда, множество последовательностей $\sim [0, 1) \sqcup \{\frac{n}{2^{k}}\ssep n, k\in \mathbb{N}\}$. Заметим, что второе множество счётно, значит множество последовательностей равномощну интервалу, который равномощен отрезку.
    \end{proof}
\end{theorem}
\begin{theorem} \thmslashn

    Множества $[0, 1]$ равномощно множеству $[0, 1] \times [0, 1]$.

    \begin{proof} \thmslashn
    
        Возьмём по предыдущей теореме последовательности $a_{i}$ и $b_{i}$ для $(a, b)\in [0, 1] \times [0, 1]$.

        Сделаем из них последовательность $a_1, b_1, a_2, b_{2}, \ldots$. Между такими последовательностями есть биекция, при этом, между второй последовательностью и отрезком есть биекция.
    \end{proof}
\end{theorem}
\begin{theorem}[Теорема Кантора] \thmslashn

    Множество бесконечных последовательностей из $\{0, 1\} $ несчётно.
    \begin{proof} \thmslashn
    
        Предположим существование биекции $f$.

        Построим последовательность $b$:
        
        Пусть $b_{1} = 1-a_{11}$, тогда $b \neq f(1)$

        Аналогично, $b_{i} \neq a_{ii}$. Тогда $\forall{i}\quad b \neq f(i)$. Противоречие.
    \end{proof}
\end{theorem}

