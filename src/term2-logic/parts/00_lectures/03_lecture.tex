\SectionLecture{Лекция 3}{Игорь Энгель}
\begin{definition} \thmslashn 

    $\le$ называется отношением частичного порядка если
    \begin{enumerate}
        \item $\forall{a}\quad a \le a$
        \item $\forall{a, b}\quad \begin{cases}
            a \le b\\
            b \le a
        \end{cases} \implies a = b$
        \item $\forall{a, b, c}\quad \begin{cases}
            a \le b\\
            b \le c
        \end{cases} \implies a \le c$ 
    \end{enumerate}

    Если $\forall{a, b}\quad a \le b \lor b \le a$, то $\le$ также называется отношение линейного порядка. 
\end{definition}
\begin{definition} \thmslashn 

    Пара $\left<X, \le \right>$ из множества и отношения частичного порядка называется частично упорядоченным множеством.
\end{definition}
\begin{definition} \thmslashn 

    $<$ называется отношением строгого частичного порядка если
    \begin{enumerate}
        \item $\nexists{a}\quad a < a$
        \item $\forall{a, b, c}\quad \begin{cases}
            a < b\\
            b < c
        \end{cases} \implies a < c$ 
    \end{enumerate}
\end{definition}
\begin{lemma} \thmslashn

    Если $\le$ - отношение частичного гопорядка, то $(\le \land x \neq y)$ - отношение строгого частичного порядка.

    Если $<$ - отношение строгого порядка, то $\left(< \lor x = y \right) $ - отношение частичного порядка.
\end{lemma}

\begin{lemma} \thmslashn

    Если $X$ - ЧУМ, то $Y \subset X$ - ЧУМ, с отношением порядка полученным ограничением отношения из $X$ на $Y$.

    Если $X, Y$ - ЧУМ, то $X \sqcup Y$ - ЧУМ.

    Если $X, Y$ - ЧУМ, то $X + Y$ - ЧУМ, такой, что $\forall{x\in X, y\in Y}\quad x \le  y$.

    Если $X, Y$ - ЧУМ, то $X \times Y$ - ЧУМ.

    Покоординатный порядок: $(x_1, y_1) \le (x_2, y_2) \iff x_1 \le x_2 \land y_1 \le y_2$.

    Лексикографический порядок: $(x_1, y_1) \le (x_2, y_2) \iff x_1 \le x_2 \lor (x_1 = x_2 \land y_1 \le y_2)$.
\end{lemma}
\begin{definition} \thmslashn 

    Элемент $x\in X$ называется наибольшим, если $\forall{y\in X}\quad y \le x$.

    Аналогично наименьший.
\end{definition}
\begin{definition} \thmslashn 

    Элемент $x\in X$ называется максимальным, если $\forall{y\in X}\quad x \le y \implies x = y$.

    Аналогично минимальный.
\end{definition}
\begin{statement} \thmslashn

    Наибольший элемент максимален.
\end{statement}
\begin{statement} \thmslashn

    Обратное в общем случае неверно.
    \begin{proof} \thmslashn
    
        \[ X = \{(1, 0), (0, 1)\}  .\]

        В покоординатном порядке эти пары не сравнимы, значит, они обе максимальны, но среди нех нет наибольшей.
    \end{proof}
\end{statement}
\begin{definition} \thmslashn 

    Пусть $\left<X, \le_{X}\right>$, $\left<Y, \le_{Y}\right>$ - ЧУМ.

    $X$ и $Y$ называются изоформными, если $\exists{f : X \mapsto Y}$, $f$ - биекция, $\forall{a, b\in X}\quad a \le_{X} b \iff f(a) \le_{Y} f(b)$.
\end{definition}
\begin{theorem} \thmslashn

    Конечные линейно-упорядоченные множества равной мощности изоморфны.

    \begin{proof} \thmslashn
    
        Индукция по мощности. Для $\emptyset$ тривиально.

        Возьмём $x_1\in X$. Либо $x_1$ наименьший, либо можем взять элемент меньше него.

        Будем выбирать $x_{i+1} < x_{i}$.

        Так-как порядок линейный, а множество конечно, то когда-нибудь придём к наименьшему элементу.

        Пусть $x_{i}$ - наименьший элемент $X$, $y_{j}$ - наименьший элемент $Y$.

        Тогда, переводим $x_{i} \to y_{i}$, дальше по индукции.
    \end{proof}
\end{theorem}
\begin{statement} \thmslashn

    Отрезок $[0, 1]$ и $\mathbb{R}$ не изоформны.
    \begin{proof} \thmslashn
    
        $1$ - наибольший элемент $[0, 1]$, а в $\mathbb{R}$ нет. Но если изоморфны, то должно быть $\forall{x\in \mathbb{R}}\quad x \le f(1)$.
    \end{proof}
\end{statement}
\begin{statement} \thmslashn

    $\mathbb{Q}$ и $\mathbb{Z}$ не изоморфны.
    \begin{proof} \thmslashn
    
        Рассмотрим $f^{-1}(1)$ и $f^{-1}(2)$. В $\mathbb{Q}$ между ними есть некое $z$. Но тогда $1 < f(z) < 2$. А такого в $\mathbb{Z}$ нет.
    \end{proof}
\end{statement}
\begin{definition} \thmslashn 

    $x$ и $y$ называются соседними элементами, если $\nexists{z}\quad x < z < y$.
\end{definition}
\begin{definition} \thmslashn 

    Порядок называется плотным, если не существует соседних элементов.
\end{definition}
\begin{theorem} \thmslashn

    Любые равномощные множетсва с плотным порядком изоморфны.
    \begin{proof} \thmslashn
    
        \TODO
    \end{proof}
\end{theorem}
\begin{definition} \thmslashn 

    Антицепь - подмножество ЧУМ элементов, где каждая пара различных элементов несравнима.
\end{definition}
\begin{definition} \thmslashn 

    Цепь - подмножество ЧУМ элементов, где каждая пара элементов сравнима.
\end{definition}
\begin{theorem}[Теорема Дилуорса] \thmslashn

    Если $\left<X, \le \right>$ - конечное ЧУМ, то размер наибольшей антицепи равен наименьшему количеству цепей, покрывающих $X$.
    \begin{proof} \thmslashn
    
       Кол-во цепей $\ge $ макс антицепь - любые элементы в антицепи нсравнимы, значит не могут лежать в одной цепи.

       Кол-во цепей $\le $ макс антицепь:

       Индукция по $|X|$: для $|X| = 0$ и $|X| =1$ очевидно.

       Выберем $m$ - минимальный элемент в $X$.

       Рассмотрим $X \setminus \{m\} $. Пусть в $X$ есть антицепь размера $S$. Если мы можем покрыть $X \setminus m$ $S - 1$ цепью, то утверждение доказано. 

       Мы знаем, что $X \setminus m$ покрывается $S$ цепями, по предположению индукции.

       Рассмотрим множество антицепей размера $S$. Каждая антицепь имеет по одному элементу с каждой цепи.

       Выберем $x_{i}$ - наименьший элемент с $i$-й цепи, который входит в хотя-бы одну антицепь размера $S$.

       Тогда $\{x_{i}\} $ образует антицепь. Если $x_1 < x_2$, рассмотрим антицепь $A$, в которую входит $x_1$. В $A$ существует $y$, находящийся в то-же цепи что $x_2$. Тогда $y > x_2 > x_1 \implies y > x_1$, что невозможно в $A$.

        Добавим $m$.  У нас всё ещё нет антицепей размера $S + 1$. Значит,  $m$ сравним с одним из  $x_{i}$. Так-как $m$ - минимальный элемент  $X$, то  $m < x_1$. Построим цепь, состоящую из куска цепи где был $x_1$, и $m$. Остался <<хвост>> цепи, на котором нет ни одного элемента входящего в антицепь размера  $S$. Тогда, у множества без новой цепи нет цепей размера  $S$, можем покрыть его  $S-1$ цепью, значит можем покрыть  $X$  $S$ цепями. 
   \end{proof}
\end{theorem}
