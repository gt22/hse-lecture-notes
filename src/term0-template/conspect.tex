% Содержит кучу стандартных настроек.
% Настоятельно рекомендуется для использования

../term0-template/core.tex

\enablecode % Включает поддержку кода.
%\lstset {
%  style=supercpp% uncomment to use c++ everywhere.
%}
\lstset {
  texcl=true %Включить, если хотите писать русские комментарии в коде.
}
% \enablemath % добавляет новые определения, например: \N, \divby

\begin{document}
% \newcommand\CustomTitle{Здесь можно создать кастомные заголовки в шапке}
\gdef\CourseName{Example course} % Обязательно
\author{Example author, and the other author} % Обязательно
% \gdef\ShortCourseName{Алгебра} % Для другого название в шапке, не обязательно
% \gdef\LaconicFooter{YES} % Минималистичный футер, только номера страниц.
% \gdef\NoTitlePage{YES} % Отключить главную страницу.

\makegood

% Если конспект очень большой, то осмысленно разбить его на куски.
% Можно сохранить отдельную часть в отдельный .tex,
% а затем написать \input{part}, просто подставить в данное место part.tex

\Section{Intro}{Example author}
  \begin{itemize}
  \item Это
  \item Секция
  \item Которая
  \item Называется
  \item Intro
  \item Can you hear me??
\end{itemize}

  
А здесь у нас ничего нет, потому что дальше идёт следующая глава...

\skipsection[2] % Пропустить две секции, обновить только нумерацию.

\Section{Привет мир}{Other author}
Для понимания языка программирования очень важно уметь писать hello world
\Subsection{Cобственно код}
\begin{lstlisting}[style=supercpp]
#include <iostream>

using namespace std;

int main() {
    vector<int> arr = {1, 2}; // работает в c++11
    cout << "Hello world\n";
    return 0;
}
\end{lstlisting}
\end{document}
