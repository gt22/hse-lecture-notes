\Subsection{Вещественные числа}
    \begin{definition} 
    Вещественные числа - множество $\mathbb{R}$, на котором задааны операции $+, \cdot : \mathbb{R} \mapsto \mathbb{R}$ и отношение $\le \subset \mathbb{R} \times \mathbb{R}$ удовлетвояющие следующим аксиомам:
        \begin{enumerate}
            \item[A1] $ a+b=b+a$
            \item[A2] $ \left( a+b \right) +c = a+\left( b+c \right)$
            \item[A3] $ \exists 0 \in \mathbb{R}\quad a + 0 = a$
            \item[A4] $\exists -a\quad a+(-a)=0$
            \item[M1] $ a\times b = b\times a$
            \item[M2] $ \left( a\times b \right) \times c = a \times \left( b \times c \right)$
            \item[M3] $ \exists 1\neq 0\quad a\times 1 = a$
            \item[M4] $ \forall{a \neq 0}\quad\exists a^{-1}\quad a\times a^{-1} = 1$
            \item[AM] $(a+b)\times c = a\times c+b\times c$
            \item[O1] $\le$ - рефлексивное антисимметричное транзитивное отношение.
            \item[O2] $\forall{a, b \in \mathbb{R}}\quad (a\le b)\lor (b\le a)$
            \item[OA] $a\le b \implies a + c \le b + c$
            \item[OM] $0 \le a, 0 \le b \implies 0 \le ab$
            \item[CA] Аксиома полноты:\\
                $\forall{\emptyset \neq A,B \subset \mathbb{R}}\quad \left( \forall{a\in A, b\in B}\quad a \le b \right) \implies \exists{c\in \mathbb{R}}\quad \forall{a\in A, b\in B}\quad a \le c \le b  $
        \end{enumerate}
    \end{definition}
    \begin{theorem}[Принцип Архимеда] 
        \[ \forall{x \in \mathbb{R}, y \in \mathbb{R}_+}\quad \exists{n \in \mathbb{N}}\quad ny>x .\]
        \begin{proof}
            \[ A = \{u \in \mathbb{R}\ssep \exists{n \in \mathbb{N}}\quad ny>u \}  .\]
            \[ 0 \in A \implies A \neq  \emptyset .\]
            
            Докажем от противного, что $A = \mathbb{R}$.

            Если $A \neq \mathbb{R}$, то $B = \mathbb{R}\setminus A \neq \emptyset$.
            \[ \forall{a \in A}\quad \forall{b \in B} \quad b\le a \implies b \in A .\] 
            \[ \forall{a \in A}\quad \forall{b \in B}\quad a\le b .\]
           
            По аксиоме полноты:
            \[ \exists{c \in \mathbb{R}}\quad \forall{a \in A}\quad \forall{b \in B}\quad a \le c \le b .\]
            
            Но $A\cup B = \mathbb{R}$, значит $\left( c \in A  \right)\lor\left( c \in B \right)  $
            
            Если $c \in A$, то $\exists{n \in \mathbb{R}}\quad c<ny \implies c+y< (n+1)y \implies c+y \in A$. Но $c+y>c$.
            
            Если  $c \in B$, то $\forall{c'<c }\quad c' \in A \implies c-y \in A \implies \exists{n \in \mathbb{R}}\quad c-y < ny \implies c<(n+1)y \implies c \in A$. Что невозможно. Значит наше предположение что  $A \neq \mathbb{R}$ неверно.
        \end{proof}
    \end{theorem}
    \begin{lemma} 
        $\forall{\varepsilon > 0}\quad \exists{n \in \mathbb{N}}\quad \frac{1}{n}<\varepsilon$.
        \begin{proof}
            Подставим $x=1$,  $y=\varepsilon$ в принцип архимеда. Тогда  $\exists{n \in \mathbb{N}}\quad n\varepsilon > 1$
        \end{proof}
    \end{lemma}
    \begin{definition} \slashns 

        $A$ ограниченно сверху, если $\exists{b \in \mathbb{R}}\quad \forall{a \in A}\quad a\le b$.

        $b$ называется верхней гранью  $A$.
    \end{definition}
    \begin{definition} \slashns
        
        $A$ ограниченно снизу, если $\exists{b \in \mathbb{R}}\quad \forall{a \in A}\quad b\le a$.

        $b$ называется нижней гранью
    \end{definition}
    \begin{lemma}
        $\mathbb{N}$ неограниченна сверху.
        \begin{proof}
            Пусть $b$ - верхняя грань  $\mathbb{N}$.
            
            Подставим $x=b$,  $y=1$ в принцип Архимеда. 
            
            Тогда $\exists{n \in \mathbb{N}}\quad b<n$.
            
            Значит $b$ -  не верхняя грань.
        \end{proof}
    \end{lemma}
    \begin{definition}[Принцип мат. индукции] \slashns

        Пусть  $P_n, n \in \mathbb{N}$ -  набор утверждений.

        Если $P_1$ верно, и  $P_n \implies P_{n+1}$, то $P_n$ верно для всех  $n$.
    \end{definition}
    \begin{theorem}
        В любом непустом конечном множестве $A$ есть наибольший и наименьший элемент.
        \begin{proof}
            Индукция по количеству элементов.

            База: Для $n=1$ - единственный элемент является наибольшим и наименьшим.

            Переход: Возьмём $n+1$-элементное множество  $A'$. Возьмём из него подмножество мощньностью  $n$, без элемента $x_{n+1}$. В этом подмножестве есть наименьшеий элемент  $x_k$. Если $x_{n+1}<x_k$ - $x_k$ наименьший элемент  $A'$, если  $x_{n+1}>x_k$ - $x_{n+1}$ - наименьший элемент $A'$. Для наибольшего симметрично.
        \end{proof}
    \end{theorem}
    \begin{lemma}
        Если $A \subset \mathbb{Z}$, $A\neq \emptyset$ и $A$ ограниченно сверху, то в  $A$ есть наибольший элемент.\\
        Если  $A \subset \mathbb{Z}$, $A \neq  \emptyset$ и $A$ ограниченно снизу, то в  $A$ есть наименьший элемент.
         \begin{proof}
            Докажем первое утверждение.
            
            $\exists{b \in \mathbb{R}}\quad \forall{a \in A}\quad a\le b$.

            Возьмём $c\in A$. Пусть $B = \{x\in A\ssep x \ge c\} $.

            Докажем что $B$ - конечное множество.

            Пусть $n=b-c$. Тогда, $|B| \le  n$, так-как в  $B$ нет элементов $>b$, и нет элементов $<c$.

            Пусть $d$ - наибольший элемент $B$. Тогда $c\le d \implies \forall{x\in A\setminus B}\quad x \le  d$.
            \begin{equation*}
                \begin{cases}
                    \forall{x\in B}\quad x \le d\\
                    \forall{x\in A\setminus B}\quad x \le d
                \end{cases} \implies \forall{x\in A}\quad x \le d
            \end{equation*}

            Значит $d$ - наибольший элемент  $A$.
        \end{proof}
    \end{lemma}
    \begin{definition}
        Целая часть числа $x \in \mathbb{R}$ ($\left\lfloor x \right\rfloor$) - наибольшее $x' \in \{x'\in \mathbb{Z}\ssep x' \le  x\} $. Такое число существует, как следствие из предыдущей леммы.
    \end{definition}
    \begin{lemma} $\left\lfloor x \right\rfloor \le x < \left\lfloor x\right\rfloor + 1$. \end{lemma}
    \begin{lemma} $x-1 < \left\lfloor x\right\rfloor \le x$. \end{lemma}
    \begin{theorem}
        Если $x, y \in \mathbb{R}$, $x<y$, то  $\exists{c \in \mathbb{Q}}\quad x<c<y$, и $\exists{c \not\in \mathbb{Q}}\quad x<c<y$
        \begin{proof}
            Пусть $\varepsilon = y-x > 0$.
            \[ \exists{n \in \mathbb{N}}\quad \frac{1}{n}<\varepsilon .\]
            \[ m = \left\lfloor nx\right\rfloor \implies r=\frac{m+1}{n} .\]

            $r$ - подходящие число, так-как  $x < \frac{m+1}{n} < y \iff  nx <  \left\lfloor nx\right\rfloor + 1 < n(x + \varepsilon)$.
            \[ \frac{1}{n} < \varepsilon \implies n\varepsilon > 1 .\] 
            
            Докажем второе утверждение: $x<y \implies x-\sqrt{2}<y-\sqrt{2}  $, тогда $\exists{r \in \mathbb{Q}}\quad x-\sqrt{2} < r < y-\sqrt{2}  $. Тогда $x<r+\sqrt{2}<y $. $r$ - рациональное, значит  $r+\sqrt{2} $ -  иррациональное.
        \end{proof}
    \end{theorem}
