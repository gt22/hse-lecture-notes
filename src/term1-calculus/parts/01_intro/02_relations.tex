\Subsection{Бинарные отношения}
    \begin{definition} 
        $R$ называется бинарным отношением между $A$ и $B$ если $R \subset \left( A \times B \right) $.

        Вхождение пары $\left<x,y\right>$ в $R$ обозначается $xRy$.
    \end{definition}
    \begin{definition} 
        Областью определения отношения $R \subset \left( A \times B \right) $ называется 
        \[\delta_{R} = \{a \ssep \exists{b\in B}\quad aRb\}\]
    \end{definition}
    \begin{definition} 
        Областью значений отношения $R \subset \left( A \times B \right) $ называется
        \[ \rho_{R} = \{b \ssep \exists{a\in A}\quad aRb\}  .\] 
    \end{definition}
    \begin{definition}  
        Обратным к отношению $R \subset \left( A \times B \right) $ называется отношение $R^{-1} \subset \left( B \times A \right) $
        \[ R^{-1} = \{\left<b, a\right> \ssep \left<a, b\right>\in R\}  .\] 
    \end{definition}
    \begin{definition} 
        Композицией отношений $R_1 \subset \left( A \times B \right) $ и $R_2 \subset \left( B \times C \right) $ называется $R_1 \circ R_2 \subset \left( A \times C \right) $
        \[ R_1 \circ R_2 = \{\left<a, c\right> | \exists{a\in A, b\in B, c\in C}\quad aR_1b \text{ и } bR_2c \}  .\] 
    \end{definition}
    \begin{example} \thmslashn 
       \begin{enumerate}
            \item $A = B$ \\
                Отношение равенства: $R = \{\left<x, x\right>\ssep x \in A\} $ 
            \item $A = B=\mathbb{R}$\\
                Отношение "$\le$": $R = \{\left<x, y\right>\ssep x \le y\}$ 
            \item $A = B = \mathbb{N}$
                Отношение "$<$":  $R = \{\left<x, y\right>\ssep x < y\} $ \\
                $\delta_R = \mathbb{N}$\\
                $\rho_R = \mathbb{N}\setminus \{1\} $\\
                $(< \circ <) = \{\left<x, z\right>\ssep x < z-1\} $
            \item Отношение перпендикулярности:\\
                $\perp \circ \perp \iff \parallel$
            \item $A$ - множество всех живших мужчин\\
                $R = \{\left<x, y\right> \ssep x \text{ отец } y\} $ \\
                $\delta_R =$ все у кого есть отец\\
                $\rho_R =$ все у кого есть сыновья\\
                $R \circ R = $ дед по отцовской линии
       \end{enumerate} 
    \end{example}
    \begin{definition} 
        Функцией $f : A \mapsto B$ называется отношение $f \subset \left( A \times B \right) $ удовлетворяющие следующим условиям:
        \begin{enumerate}
            \item $\delta_f = A$
            \item $\forall{a \in A, b, b'\in B}\quad \left( afb \text{ и } afb' \right) \implies b=b' $
        \end{enumerate}
        Если $f$ - функция, то вхождение пары $\left<a, b\right>$ в $f$ обозначается как $b = f(a)$.
    \end{definition}
    \pagebreak
    \begin{properties} 
        Пусть $R \subset A \times A$.

        $R$ - рефлексивное елси $\forall{a\in A}\quad aRa$

        $R$ - иррефлексивное если $\nexists{a \in A}\quad aRa$

        $R$ - симметричное если  $\forall{a, a' \in A}\quad aRa' \iff a'Ra$

        $R$ - антисимметричное если  $\forall{a, a' \in A}\quad \left(aRa' \iff  a'Ra\right) \implies a=a'$

        $R$ - транзитивное если $\forall{a, a', a'' \in A}\quad aRa'\land a'Ra'' \implies aRa''$

        Для определённых наборов свойств есть свои названия:

        $R$ - отношение строгого частичного порядка, если оно иррефлексивно, антисимметрично и транзитивно.

        $R$ - отношение нестрогого частичного порядка, если оно рефлексивно, антисиимметрично и транзитивно.
        
        $R$ - отношение эквивалентности, если оно рефлексивно, симметрично и транзитивно.
    \end{properties}
    \begin{properties} 
        Пусть $f : A \mapsto B$

        $f$ - инъективная, если  $\forall{x, y \in A}\quad f(x) = f(y) \implies x = y$

        $f$ - сюръективна, если  $\rho_f = B$

        $f$ - биективна, если она инъективна и суръективна
    \end{properties}
    \begin{example} \thmslashn 
        \begin{enumerate}
            \item Равенство, сравнение по модулю, паралелльность, подобие треугольников - отношения эквивалентсноти
            \item Отношения нестрогого частичного порядка:\\
            $A=\mathbb{R}, R = \le$\\
            $A=2^C,  R = \subset$
            \item Отношения строгого частичного порядка\\
            $A=\mathbb{R}, R = <$\\
            $A=2^C, R = \subsetneq$
        \end{enumerate}  
    \end{example}
