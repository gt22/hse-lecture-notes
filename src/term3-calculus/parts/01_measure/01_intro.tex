\Subsection{Напоминания (из прошлого семестра)}
\def\questionsurl{https://github.com/gt22/hse-questions-calculus/blob/pdf/term2-calculus-questions.pdf}

\begin{definition}[Алгебра множеств] \thmslashn 

    Пусть $X$ - множество. $\mathcal{A} \subset 2^{X}$ называется алгеброй, если
    
    \[ \emptyset\in \mathcal{A} .\] 
    \[ A\in \mathcal{A} \implies X \setminus A\in \mathcal{A} .\] 

    \begin{equation*}
        A, B\in \mathcal{A} \implies
        \begin{cases}
            A\cap B\in \mathcal{A}\\
            A \cup B\in \mathcal{A}
        \end{cases}
    \end{equation*}

    И называется $\sigma$-алгеброй, если ещё
    \begin{equation*}
        A_{n}\in \mathcal{A} \implies
        \begin{cases}
            \bigcap\limits_{n=1}^{\infty} A_{n}\in \mathcal{A}\\
            \bigcup\limits_{n=1}^{\infty} A_{n}\in \mathcal{A}
        \end{cases}
    \end{equation*}
\end{definition}
\begin{definition}[Кольцо] \thmslashn 

    $\mathcal{R} \subset 2^{X}$ называется кольцом, если
    \begin{equation*}
        A, B\in \mathcal{R} \implies
        \begin{cases}
            A\cap B\in \mathcal{R}\\
            A \cup B\in \mathcal{R}\\
            A \setminus B\in \mathcal{R}
        \end{cases}
    \end{equation*}
\end{definition}
\begin{lemma} \thmslashn

    \begin{equation*}
        \begin{cases}
            \mathcal{R} \text{ - кольцо над $X$}\\
            X\in \mathcal{R}
        \end{cases} \implies \mathcal{R} \text{ - алгебра над $X$}
    \end{equation*}
    \begin{proof} \thmslashn
    
        \[ X\in \mathcal{R} \implies X \setminus X = \emptyset\in \mathcal{R} .\]
        \[ X, A\in \mathcal{R} \implies X \setminus A\in \mathcal{R} .\]
        \begin{equation*}
            A, B\in \mathcal{R} \implies
            \begin{cases}
                A\cap B\in \mathcal{R}\\
                A \cup B\in \mathcal{R}
            \end{cases} \qedhere
        \end{equation*}
    \end{proof}
\end{lemma}
\begin{definition}[Полукольцо] \thmslashn 

    $\mathcal{P} \subset 2^{X}$ называется полукольцом, если
    \[ \emptyset\in \mathcal{P} .\]
    \begin{equation*}
        A, B\in \mathcal{P} \implies
        \begin{cases}
            A\cap B\in \mathcal{P}\\
            \exists{Q_{n}\in \mathcal{P}}\quad A \setminus B = \bigsqcup\limits_{k=1}^{n} Q_{k}
        \end{cases}
    \end{equation*}
\end{definition}
\begin{theorem}[Разность и объединение нескольких множеств в полукольце]\label{thm:semi_ops} \thmslashn

    \[ P_{n}, P\in \mathcal{P} \implies \exists{Q_{m}\in \mathcal{P}}\quad P \setminus \bigcup\limits_{k=1}^{n} P_{k} = \bigsqcup\limits_{k=1}^{m} Q_{k}  .\]

    \begin{equation*}
        P_{k} \in \mathcal{P} \implies \exists{Q_{jk}\in \mathcal{P}}\quad
        \begin{cases}
            Q_{jk} \subset P_{k}\\
            \bigcup\limits_{k=1}^{n} P_{k} = \bigsqcup\limits_{j=1}^{n} \bigsqcup\limits_{k=1}^{m_{j}} Q_{jk} 
        \end{cases}
    \end{equation*}

    \begin{equation*}
        P_{n}\in \mathcal{P} \implies \exists{Q_{nk}\in \mathcal{P}}\quad
        \begin{cases}
            Q_{nk} \subset P_{n}\\
            \bigsqcup\limits_{n=1}^{\infty} \bigsqcup\limits_{k=1}^{m_{n}} Q_{nk} 
        \end{cases}
    \end{equation*}
    \begin{proof} \thmslashn
    
        \href{\questionsurl}{В билете 100}
    \end{proof}
\end{theorem}
\begin{definition}[Ячейка] \thmslashn 

    Ячейка на векторах $a, b\in \R^{n}$ $[a, b) := [a_1, b_1) \times [a_2, b_2) \times \ldots \times [a_{n}, b_{n})$
\end{definition}
\begin{theorem} \thmslashn

    Если $G \subset \R^{m}$ - непустое открытое множество, ето 

    \[ \exists{a_{k}, b_{k}\in \Q^{m} \subset \R^{m}}\quad G = \bigsqcup\limits_{k=1}^{n} [a_{k}, b_{k})    .\] 
    \begin{proof} \thmslashn
    
        \href{\questionsurl}{В билете 102}
    \end{proof}
\end{theorem}



\undef\questionsurl
