\Subsection{Объём и мера}
\begin{definition} \thmslashn 

    Пусть $\mathcal{P} \subset 2^{X}$ - полукольцо.

    $\mu : \mathcal{P} \to [0, \infty]$ называется объёмом, если (при условии, что все эти множества лежат в $\mathcal{P}$)

    \[ \mu \emptyset = 0  .\]
    \[ \mu\left( \bigsqcup\limits_{k=1}^{n} A_{k} \right) = \sum\limits_{k=1}^{n} \mu A_{k}  .\] 

    и называется мерой, если ещё
    
    \[ \mu\left( \bigsqcup\limits_{k=1}^{\infty} A_{k} \right) = \sum\limits_{k=1}^{\infty} \mu A_{k}  .\] 
\end{definition}
\begin{lemma} \thmslashn

    Первое условие можно заменить на $\exists{x\in \mathcal{P}}\quad \mu x \neq  +\infty$.
    \begin{proof} \thmslashn
    
        \[ \mu x = \mu (x \sqcup \emptyset) = \mu x + \mu \emptyset \implies \bcases{\mu x = +\infty\cr \mu \emptyset = 0} .\qedhere\] 
    \end{proof}
\end{lemma}
\begin{example}[объёма] \thmslashn

    \begin{enumerate}
        \item Длина на ячейках в $\R$
        \item $g : \R \to \R$ - неубывающая функция. $\mu_{g}([a, b)) = g(b) - g(a)$
        \item $\lambda_{m}$ - объём на ячейках в $\R^{m}$
        \item $x_0\in X$, $\mu A := \begin{cases}
                a & x_0\in A\\
                0 & x_0 \not\in A
        \end{cases}$
    \item $\mathcal{A}$ - алгебра ограниченных множеств и их дополнений в  $\R^{m}$, $A\in \mathcal{A}$. $\mu A := \begin{cases}
            0 & A  \text{ ограничено}\\
            1 & A \text{ неограниченное}
    \end{cases}$. $\mu$ - объём, но не мера.
    \end{enumerate}
\end{example}
\begin{properties} \thmslashn

    Пуcть $\mu$ - объём на полукольце $\mathcal{P}$. Тогда

    \begin{enumerate}
        \item $P' \subset P \implies \mu P' \le \mu P$
        \item $\bigsqcup\limits_{k=1}^{n} P_{k} \subset P \implies \sum\limits_{k=1}^{n} \mu P_{k} \le \mu P$
            \begin{proof} \thmslashn
            
                Рассмотрим $P \setminus \bigsqcup\limits_{k=1}^{n} P_{k}= \bigsqcup\limits_{k=1}^{m} Q_{k} $ (\autoref{thm:semi_ops}).

                Тогда $P = \bigsqcup\limits_{k=1}^{n} P_{k}\sqcup \bigsqcup\limits_{k=1}^{m} Q_{k} \implies \mu P = \sum\limits_{k=1}^{n} \mu P_{k} + \sum\limits_{k=1}^{m} \mu Q_{k} \ge \sum\limits_{k=1}^{n} P_{k}  $
            \end{proof}
        \item[2'] $\bigsqcup\limits_{k=1}^{\infty} P_{k} \subset P \implies \sum\limits_{k=1}^{\infty} \mu P_{k} \le \mu P$
            \begin{proof} \thmslashn
            
                Рассмотрим частичные объединеиня, получим что 
                \[ \sum\limits_{k=1}^{n} \mu P_{k} \le \mu P .\] 

                Устремим $n \to \infty$, получим неравенство.
            \end{proof}
        \item $P \subset \bigcup\limits_{k=1}^{n} P_{k} \implies \mu P \le \sum\limits_{k=1}^{n} \mu P_{k}$
            \begin{proof} \thmslashn
            
                Пусть $P_{k}' := P\cap P_{k}$. 
                
                \begin{equation*}
                    \begin{cases}
                        P_{k}'\in \mathcal{P}\\
                        P = \bigcup\limits_{k=1}^{n} P_{k}'
                    \end{cases} \overset{\text{\autoref{thm:semi_ops}}}{\implies} 
                    P = \bigsqcup\limits_{j=1}^{n}\bigsqcup\limits_{k=1}^{m_{j}} Q_{jk}
                    \implies \mu P = \sum\limits_{j=1}^{n}\sum\limits_{k=1}^{m_{j}} \mu Q_{jk}
                \end{equation*}


                При этом, $Q_{jk} \subset P_{j}' \implies \bigsqcup\limits_{k=1}^{m_{j}} Q_{jk} \subset P_{j}'$.

                Тогда $\mu P \le \sum\limits_{j=1}^{n} \mu P_{j}' \le \sum\limits_{k=1}^{n} \mu P_{j}$
            \end{proof}
    \end{enumerate}
\end{properties}
\begin{remark} \thmslashn

    Если $\mathcal{P}$ - алгебра, $A, B\in \mathcal{P}$, $B \subset A$, $\mu B < +\infty$, то $\mu (A \setminus B) = \mu A - \mu B$. 
\end{remark}
\begin{remark} \thmslashn

    Объём с полукольца можно продолжить на кольцо.

    Построим кольцо из всех возможных $\bigsqcup$ полукольца. На них можем однозначно задать объём как сумму.
\end{remark}
\begin{theorem} \thmslashn

    Пусть $\mathcal{P}$ и $\mathcal{Q}$ - полукольца на $X$ и $Y$, $\mu, \nu$ - объёмы на них.

    Тогда $\lambda$ - объём на $\mathcal{P} \times \mathcal{Q}$:
    \[ \lambda(P \times Q) := \mu P \cdot \nu Q \text{ (считая $+\infty \cdot 0 = 0$ )}.\]

    \begin{proof} \thmslashn
   
        Предположим, что получилось найти разложения
        \[ P = \bigsqcup\limits_{k=1}^{n} P_{k} .\]
        \[ Q = \bigsqcup\limits_{j=1}^{m} Q_{j} .\]

        Такие, что 
        \[ P \times Q = \bigsqcup\limits_{k=1}^{n} \bigsqcup\limits_{j=1}^{m} P_{k} \times Q_{j} .\] 

        Тогда,
       \[ \lambda(P \times Q) = \mu P \cdot \nu Q = \left(\sum\limits_{k=1}^{n} P_{k}\right) \cdot \left( \sum\limits_{j=1}^{m} Q_{j} \right) = \sum\limits_{k=1}^{n} \sum\limits_{j=1}^{m} \lambda(P_{k} \cdot Q_{j}).\]

       Может сулчиться, что таких разбиений не найдётся,
       
       тогда \sout{(истерически размахивая руками)} разделим всё на мелкие кусочки,
       
       получим разбиение для множеств, а на кусочках посчитаем рекурсивно. Утверждается, что когда-то сойдётся.
    \end{proof}
\end{theorem}
\begin{consequence} \thmslashn

    \[ \lambda_{m} [a, b) = (b_1 - a_1)(b_2 - a_2) \ldots (b_{m} - a_{m}) .\]

    $\lambda_{m}$ - действительно объём на ячейках в $\R^{m}$
\end{consequence}
\begin{example}[меры] \thmslashn

    \begin{enumerate}
        \item $\lambda_{m}$ (классический объём) на ячейках
        \item $g : \R \mapsto \R$ непрерывная слева неубывающая функция. $\mu_{g}([a, b)) = g(b) - g(a)$.
        \item $x_0\in X$, $\mu A := \begin{cases}
                a & x_0\in A\\
                0 & x_0 \not\in A
        \end{cases}$
    \item Считающая мера - $\mu A = |A|$
    \item $T = \{t_1, t_2, \ldots\} \subset X $ - не более чем счётное, сопоставим им $\{w_1, w_2, \ldots\} \subset N_{\ge 0}$.\\
        $\mu A := \sum\limits_{t_{i}\in A\cap T} w_{i}$
        \begin{proof} \thmslashn
        
            Пусть $A = \bigsqcup\limits_{n=1}^{\infty} A_{n}$.

            $\mu A_{n} = \sum\limits_{k=1}^{\infty} a_{nk}$ (обозначим так те числа, что надо сложить).

            \[ \mu A = \sum\limits_{n=1}^{\infty} \sum\limits_{k=1}^{\infty} a_{nk} .\]

            Если $\exists{n}\quad t_{i}\in A_{n}$, то $\exists!{n, k}\quad t_{i} = a_{nk}$, и при этом $t_{i}\in A$. Точно также, $\forall{k}\quad a_{nk}\in A_{n} \subset A$, значит равенство верно. (тут рукамахательство с уникальностью, но вроде понятно почему работает)
        \end{proof}
    \end{enumerate}
\end{example}
\begin{theorem} \thmslashn

    Пусть $\mu$ - объём на полукольце $\mathcal{P}$.

    \begin{equation*}
        \mu \text{ - мера} \iff 
        \left(\begin{cases}
            P_{n}, P\in \mathcal{P}\\
            P \subset \bigcup\limits_{n=1}^{\infty} P_{n}
    \end{cases} \implies \mu P \le \sum\limits_{n=1}^{\infty} \mu P_{n}\right)
    \end{equation*}

    \begin{proof} \thmslashn
    
        $\impliedby$:

        Если $P = \bigsqcup\limits_{k=1}^{\infty} P_{k}$, то это условие даёт $\le $, а $\ge $ верно всегда (свойства объёма)

        $\implies$:

        $P_{n}' := P\cap P_{n}$. Тогда $P = \bigcup\limits_{k=1}^{\infty} P_{n}' \implies P = \bigsqcup\limits_{k=1}^{\infty} \bigsqcup\limits_{j=1}^{m_{k}} Q_{kj}$, $Q_{kj} \subset P_{k}'$.

        \begin{equation*}
            P = \bigcup\limits_{n=1}^{\infty} P_{n}' \overset{\text{\autoref{thm:semi_ops}}}{\implies}
            \begin{cases}
                P = \bigsqcup\limits_{n=1}^{\infty} \bigsqcup\limits_{k=1}^{m_{n}} Q_{nk}\\
                Q_{nk} \subset P_{j}'
            \end{cases}
        \end{equation*}

        Тогда $\sum\limits_{k=1}^{m_{n}} \mu Q_{nk} \le \mu P_{n}' $.

        Так-как $\mu$ - мера, то $\mu P = \sum\limits_{n=1}^{\infty} \sum\limits_{k=1}^{m_n} \mu Q_{nk} \le \sum\limits_{n=1}^{\infty} \mu P_{n}'$
    \end{proof}
\end{theorem}
\begin{consequence} \thmslashn

    Если $\mu$ - мера на $\sigma$-алгебре, то счётное объединение множеств нулевой меры имеет нулевую меру.
\end{consequence}
\begin{theorem}[Непрерывность снизу] \thmslashn

    Пусть $\mu$ - объём на $\sigma$-алгебре.

    $\mu$ - мера $\iff$ $A_1 \subset A_2 \subset \ldots \implies \mu A_{k} \to \mu\left( \bigcup\limits_{k=1}^{\infty}  A_{k}\right)$
    \begin{proof} \thmslashn
    
        $\implies$


        $A_0 := \emptyset$, $B_{k} := A_{k} \setminus A_{k-1}$. Тогда, $A := \bigcup\limits_{k=1}^{\infty} A_{k} = \bigsqcup\limits_{k=1}^{\infty} B_{k}$.

        $\mu A = \sum\limits_{k=1}^{\infty} \mu B_{k} = \lim\limits_{n \to \infty} \sum\limits_{k=1}^{n} \mu B_{k} = \lim\limits_{n \to \infty}  \mu \left( \bigsqcup\limits_{k=1}^{n} B_{k} \right) = \lim\limits_{n \to \infty} \mu A_{n}$ 

        $\impliedby$

        Пусть $A = \bigsqcup\limits_{n=1}^{\infty} C_{n}$. Возьмём $A_{n} := \bigsqcup\limits_{k=1}^{n} C_{k}$.

        Имеем $\mu A = \lim\limits_{n \to \infty} \mu A_{n} = \lim\limits_{n \to \infty} \mu \left( \bigsqcup\limits_{k=1}^{n} C_{k} \right) = \lim\limits_{n \to \infty} \sum\limits_{k=1}^{n} \mu C_{k} = \sum\limits_{n=1}^{\infty} \mu C_{n}$.
    \end{proof}
\end{theorem}
\begin{theorem} \thmslashn

    Пусть $\mu$ - объём на $\sigma$-алгебре, и $\mu X < +\infty$.

    Тогда следующие условия равносильны:
    \begin{enumerate}
        \item $\mu$ - мера
        \item $A_1 \supset A_2 \supset \ldots$, $A = \bigcap\limits_{k=1}^{\infty} A_{k} \implies \mu A_{k} \to \mu A $ (непрерывность сверху)
        \item $A_1 \supset A_2 \supset \ldots$, $\bigcap\limits_{k=1}^{\infty} A_{k} = \emptyset \implies \mu A_{k} \to 0 $.
    \end{enumerate}
    \begin{proof} \thmslashn
    
        $1 \implies 2$:

        $B := A_1 \setminus A$, $B_{k} := A_1 \setminus A_{k}$.

        Тогда $B_1 \subset B_2 \subset \ldots$, и применима предыдущая теорема:
        \begin{equation*}
            \begin{split}
                \mu A_1 - \mu A_{k}
                &= \mu(A_1 \setminus A_{k})\\
                &= \mu B_{k} \to \mu B\\
                &= \mu(A_1 \setminus A)\\
                &= \mu A_1 - \mu A\\
                &\implies \mu A_1 - \mu A_{k} \to \mu A_1 - \mu A\\
                &\implies \mu A_{k} \to \mu A 
            \end{split}
        \end{equation*}

        $2 \implies 3$ тривиально.

        $3 \implies 1$:

        $A = \bigsqcup\limits_{k=1}^{\infty} C_{k}$, $A_{n} := \bigsqcup\limits_{k=n+1}^{\infty} C_{k} $ 

        Имеем что $\mu A_{k} \to 0$

        $\mu A = \mu \left( \bigsqcup\limits_{k=1}^{n} C_{k} \sqcup A_{n} \right) = \sum\limits_{k=1}^{n} \mu C_{k} + \mu A_{n} \to \sum\limits_{k=1}^{\infty} C_{k} + 0 $
    \end{proof}
\end{theorem}
\begin{consequence} \thmslashn

    Достаточно чтобы $\mu A_{m} < +\infty$ при некотором $m$, тогда есть $2$, $3$
\end{consequence}
