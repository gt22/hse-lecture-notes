% Содержит кучу стандартных настроек.
% Настоятельно рекомендуется для использования

../term0-template/core.tex
\usepackage{import}

\enablecode % Включает поддержку кода.
%\lstset {
%  style=supercpp% uncomment to use c++ everywhere.
%}
\lstset {
  texcl=true %Включить, если хотите писать русские комментарии в коде.
}
\enablemath % добавляет новые определения, например: \N, \divby

\begin{document}
% \newcommand\CustomTitle{Здесь можно создать кастомные заголовки в шапке}
\gdef\CourseName{Математический Анализ} % Обязательно
\author{Игорь Энгель} % Обязательно
\gdef\ShortCourseName{Матан} % Для другого название в шапке, не обязательно
% \gdef\LaconicFooter{YES} % Минималистичный футер, только номера страниц.
% \gdef\NoTitlePage{YES} % Отключить главную страницу.

\makegood

% Если конспект очень большой, то осмысленно разбить его на куски.
% Можно сохранить отдельную часть в отдельный .tex,
% а затем написать \input{part}, просто подставить в данное место part.tex
\newcommand\load[1]{\import{parts/#1/}{00_main_lectures}}
\setcounter{section}{-1}
\Section{Конкспект по лекциям}{Игорь Энгель}
Это конспект сгруппированный по лекциям, потому-что так его удобнее писать. \textbf{Ошибки в этой версии конспекта не исправляются}. Этот конспект может обновляться чуть раньше основного.
\load{00_lectures}
\end{document}
