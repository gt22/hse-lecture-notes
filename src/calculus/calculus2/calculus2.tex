% !TEX encoding = UTF-8 Unicode
\documentclass[11pt, oneside]{article}   	% use "amsart" instead of "article" for AMSLaTeX format
\usepackage{amssymb}
\usepackage{amsmath}
\usepackage{cRussian}
\usepackage{cPicture}
\usepackage{cTheorem}
%\usepackage{cFonts}
%\usepackage{cTikz}
\title{Мат. Анализ 2}
\author{Igor Engel}
\date{}

\begin{document}
\maketitle
\section{Вещественные числа}
\begin{definition}
    Веществыенные числа - множество, на котором заданы операции $+, \times : \mathbb{R} \times \mathbb{R} \mapsto \mathbb{R}$, удовлетворяющие следующим аксиомам, $a, b, c \in \mathbb{R}$
    \begin{enumerate}
        \item[A1] $ a+b=b+a$
        \item[A2] $ \left( a+b \right) +c = a+\left( b+c \right)$
        \item[A3] $ \exists 0 \in \mathbb{R}\quad a + 0 = a$
        \item[A4] $\exists -a\quad a+(-a)=0$
        \item[M1] $ a\times b = b\times a$
        \item[M2] $ \left( a\times b \right) \times c = a \times \left( b \times c \right)$
        \item[M3] $ \exists 1\neq 0\quad a\times 1 = a$
        \item[M4] $ \forall{a \neq 0}\quad\exists a^{-1}\quad a\times a^{-1} = 1$
        \item[AM] $(a+b)\times c = a\times c+b\times c$
    \end{enumerate}
    А так-же обладают отношением порядка $\le \subset \mathbb{R}^2$, удовлетворяющего следующим аксомам:
    \begin{enumerate}
        \item $\le$ - рефлексивное антисимметричное транзитивное отношение.
        \item $\forall{a, b \in \mathbb{R}}\quad (a\le b)\lor (b\le a)$
        \item $a\le b \implies a + c \le b + c$
        \item $0 \le a, 0 \le b \implies 0 \le ab$
    \end{enumerate}
    И удовлетворяет аксиоме полноты:
    \[ A \subset \mathbb{R}, A \neq \emptyset .\]
    \[ B \subset \mathbb{R}, B \neq \emptyset .\] 
    \[ \forall{a \in A}\quad \forall{b \in B}\quad a \le b .\]
    \[ \exists{c \in \mathbb{R}}\quad \forall{a \in A}\quad \forall{b \in B}\quad a \le c \le b .\]
\end{definition}
\begin{theorem}[Принцип Aрхимеда]
    \[ \forall{x \in \mathbb{R}, y \in \mathbb{R}_+}\quad \exists{n \in \mathbb{N}}\quad ny>x .\]
    \begin{proof}
        \[ A = \{u \in \mathbb{R}\ssep \exists{n \in \mathbb{N}}\quad ny>u \}  .\]
        \[ 0 \in A \implies A \neq  \emptyset .\]
        Докажем от противного, что $A = \mathbb{R}$.\\
        Если $A \neq \mathbb{R}$, то $B = \mathbb{R}\setminus A \neq \emptyset$.
        \[ \forall{a \in A}\quad \forall{b \in B} \quad b\le a \implies b \in A .\] 
        \[ \forall{a \in A}\quad \forall{b \in B}\quad a\le b .\]
        По аксиоме полноты:
        \[ \exists{c \in \mathbb{R}}\quad \forall{a \in A}\quad \forall{b \in B}\quad a \le c \le b .\]
        Но $A\cup B = \mathbb{R}$, значит $\left( c \in A  \right)\lor\left( c \in B \right)  $\\ 
        Если $c \in A$, то $\exists{n \in \mathbb{R}}\quad c<ny \implies c+y< (n+1)y \implies c+y \in A$. Но $c+y>c$.\\
        Если  $c \in B$, то $\forall{c'<c }\quad c' \in A \implies c-y \in A \implies \exists{n \in \mathbb{R}}\quad c-y < ny \implies c<(n+1)y \implies c \in A$. Что невозможно. Значит наше предположение что  $A \neq \mathbb{R}$ неверно.
    \end{proof}
    \end{theorem}
    \begin{tlemma}
        \[ \forall{\varepsilon > 0}\quad \exists{n \in \mathbb{N}}\quad \frac{1}{n}<\varepsilon  .\]
        \begin{proof}
            Подставим $x=1$,  $y=\varepsilon$ в принцип архимеда. Тогда  $\exists{n \in \mathbb{N}}\quad n\varepsilon > 1$
        \end{proof}
    \end{tlemma}
    \begin{definition}
        $A$ ограниченно сверху, если 
        \[ \exists{b \in \mathbb{R}}\quad \forall{a \in A}\quad a\le b  .\].
        $b$ называется верхней гранью  $A$.
    \end{definition}
    \begin{definition}
        $A$ ограниченно снизу, если 
        \[ \exists{b \in \mathbb{R}}\quad \forall{a \in A}\quad b\le a .\]
        $b$ называется нижней гранью
    \end{definition}
    \begin{tlemma}
        $\mathbb{N}$ неограниченна сверху.\\
        \begin{proof}
            Пусть $b$ - верхняя грань  $\mathbb{N}$. Подставим $x=b$,  $y=1$ в принцип Архимеда. Тогда 
            \[ \exists{n \in \mathbb{N}}\quad b<n .\]
            Значит $b$ -  не верхняя грань.
        \end{proof}
    \end{tlemma}
    \begin{definition}[Принцип мат. индукции]
        Пусть  $P_n, n \in \mathbb{N}$ -  набор утверждений.\\
        Если $P_1$ верно, и  $P_n \implies P_{n+1}$, то $P_n$ верно для всех  $n$.
    \end{definition}
    \begin{theorem}
        В любом непустом конечном множестве $A$ есть наибольший и наименьший элемент.
        \begin{proof}
            Индукция по количеству элементов.\\
            База: Для $n=1$ - единственный элемент является наибольшим и наименьшим.\\
            Переход: Возьмём $n+1$-элементное множество  $A'$. Возьмём из него подмножество мощньностью  $n$, без элемента $x_{n+1}$. В этом подмножестве есть наименьшеий элемент  $x_k$. Если $x_{n+1}<x_k$ - $x_k$ наименьший элемент  $A'$, если  $x_{n+1}>x_k$ - $x_{n+1}$ - наименьший элемент $A'$. Для наибольшего симметрично.
        \end{proof}
    \end{theorem}
    \begin{tlemma}
        Если $A \subset \mathbb{Z}$, $A\neq \emptyset$ и $A$ ограниченно сверху, то в  $A$ есть наибольший элемент.\\
        Если  $A \subset \mathbb{Z}$, $A \neq  \emptyset$ и $A$ ограниченно снизу, то в  $A$ есть наименьший элемент.
         \begin{proof}
            Докажем первое утверждение.
            \[ \exists{b \in \mathbb{R}}\quad \forall{a \in A}\quad a\le b .\]
            Возьмём $c\in A$. Пусть $B = \{x\in A\ssep x \ge c\} $.\\ 
            Докажем что $B$ - конечное множество.
            Пусть $n=b-c$. Тогда, $|B| \le  n$, так-как в  $B$ нет элементов больше  $b$, и нет элементов меньших  $c$.\\
            Пусть $d$ - наибольший элемент $B$. Тогда $c\le d \implies \forall{x\in A\setminus B}\quad x \le  d$.\\
            \begin{equation*}
                \begin{cases}
                    \forall{x\in B}\quad x \le d\\
                    \forall{x\in A\setminus B}\quad x \le d
                \end{cases} \implies \forall{x\in A}\quad x \le d
            \end{equation*}
            Значит $d$ - наибольший элемент  $A$.
        \end{proof}
    \end{tlemma}
    \begin{definition}
        Целая часть числа $x \in \mathbb{R}$ ($\left\lfloor x \right\rfloor$) - наибольшее $x' \in \{x'\in \mathbb{Z}\ssep x' \le  x\} $. Такое число существует, как следствие из предыдущей леммы.
    \end{definition}
    \begin{dlemma}
        \[ \left\lfloor x \right\rfloor \le x < \left\lfloor x\right\rfloor + 1 .\]        
    \end{dlemma}
    \begin{dlemma}
        \[ x-1 < \left\lfloor x\right\rfloor \le x.\] 
    \end{dlemma}
    \begin{theorem}
        Если $x, y \in \mathbb{R}$, $x<y$, то  $\exists{c \in \mathbb{Q}}\quad x<c<y$, и $\exists{c \not\in \mathbb{Q}}\quad x<c<y$
        \begin{proof}
            Пусть $\varepsilon = y-x > 0$.
            \[ \exists{n \in \mathbb{N}}\quad \frac{1}{n}<\varepsilon .\]
            \[ m = \left\lfloor nx\right\rfloor \implies r=\frac{m+1}{n} .\]
            $r$ - подходящие число, так-как  $x < \frac{m+1}{n} < y \iff  nx <  \left\lfloor nx\right\rfloor + 1 < n(x + \varepsilon)$.
            \[ \frac{1}{n} < \varepsilon \implies n\varepsilon > 1 .\] 
            Докажем второе утверждение: $x<y \implies x-\sqrt{2}<y-\sqrt{2}  $, тогда $\exists{r \in \mathbb{Q}}\quad x-\sqrt{2} < r < y-\sqrt{2}  $. Тогда $x<r+\sqrt{2}<y $. $r$ - рациональное, значит  $r+\sqrt{2} $ -  иррациональное.
        \end{proof}
    \end{theorem}
\section{Супремум и инфинум}
    \begin{definition}
        $d=\sup A$ - мнимальная верхняя грань:
        \[ \forall{a \in A}\quad a\le   d.\] 
        \[\forall{\varepsilon > 0}\quad \exists{a \in A}\quad a>  (d-\varepsilon) .\] 
    \end{definition}
    \begin{definition}
        $d=\inf A$ - максимальная нижняя грань:
    \[ \forall{a \in A}\quad a\ge   d.\] 
    \[\forall{\varepsilon > 0}\quad \exists{a \in A}\quad a< (d+\varepsilon) .\] 
    \end{definition}
    \begin{theorem}
        $A \neq \emptyset$ ограниченно сверху, то существует $\sup A$.
         \begin{proof}
            $B$ - множество всех верхних граней  $A$.  $B \neq \emptyset$.
            \[ \forall{b \in B}\quad \forall{a \in A}\quad a\le b .\]
            По аксиоме полноты:
            \[ \exists{c \in \mathbb{R}}\quad \forall{a \in A}\quad \forall{b \in B}\quad a\le c\le b .\] 
            $c$ - верхняя грань, значит  $c \in B$. И $c$ - наименьший элемент  $B$.
        \end{proof}
        
    \end{theorem}
    \begin{tlemma}
        Пусть $B \neq \emptyset$, $B \subset A$, $A$ ограниченно сверху. Тогда  $\sup A \ge \sup B$.\\
        \begin{proof}
            $\sup A$ - верхняя грань  $A$ и $B$. Тогда, наименьшая верхняя грань  $B$ не больше  $\sup A$.
        \end{proof}
    \end{tlemma}
    Эти теорема и лемма симметричны для инфинума.
    \begin{definition}
        Если $A$ неограниченно сверху, то  $\sup A = \infty$.\\
        Если $A$ неограниченно снизу, то  $\inf A = -\infty$.
        \[ \forall{r \in \mathbb{R}}\quad -\infty < r < \infty .\]
        Тогда со всех предыдущих теорем можно снять условия на ограниченность множеств.
    \end{definition}
    \begin{theorem}[Теорема о вложенных отрезках]
        \[ [a_1;b_1] \supset [a_2, b_2] \supset \ldots .\]
        \[ \bigcup\limits_{i =0}^{\infty}[a_i; b_i] \neq \emptyset  .\] 
        \begin{proof}
            \[ a_1 < b_1 .\] 
            \[ a_1 \le a_2 \le \ldots .\]
            \[ b_1 \ge b_2 \ge \ldots .\]
            \[ A = \{a_k\ssep k \in N\}  .\]
            \[ B = \{b_k\ssep k \in N\}  .\]
            \[\forall{i, j \in \mathbb{N}}\quad a_i \le b_j .\]
            Тогда, по аксиоме полноты:
            \[ \exists{c \in \mathbb{R}}\quad \forall{i, j \in \mathbb{N}}\quad a_i \le  c \le b_i .\]
            \[ \forall{k \in \mathbb{N}}\quad a_k \le c \le b_k .\]
            Тогда $c \in \bigcup\limits_{i=1}^{\infty}[a_i; b_i] $, а значит оно непустое.
        \end{proof}
        Дополнительно:\\
        В $\mathbb{Q}$ теорема не верна.\\
        Для полуинтервалов и лучей теорема не верна.
    \end{theorem}        
\section{Последовательности вещественных чисел}
    Последовательность вещественных чисел: $f : \mathbb{N} \mapsto \mathbb{R}$\\
    Например, $x_n = f(n) = \sqrt{n^2+1} $.
    \subsection{Предел последовательности}
        \begin{definition}
            $x_n$ - ограниченна сверху, если
            \[ \exists{b \in \mathbb{R}}\quad \forall{n \in \mathbb{N}}\quad x_n \le b  .\] 
        \end{definition}
        \begin{definition}[Предел последовательности]
             $\ell = \lim x_n$, если для любого интервала $(a, b)$ содержащего  $\ell$, вне его лежит лишь конечное число членов последовательности.
        \end{definition}
        \begin{tlemma}
            Можно рассматривать симметричный отностиельно точки $\ell$ интервал.\\
            \begin{proof}
                Возьмём интервал  $\left( \ell-x'; \ell+x \right) $, где $x'>x$.\\
                Если вне интервала  $\left( \ell-x; \ell+x \right)$ лежит конечное число членов последовательности, то для изначального интервала это тоже верно.
            \end{proof}
        \end{tlemma}
        \begin{definition}[Эквивалентное определение предела последовательности]
        \[\ell = \lim x_n \iff \forall{\varepsilon >0}\quad \exists{n_0 \in \mathbb{N}}\quad \forall{n > n_0 \in \mathbb{N}}\quad |x_n-\ell|<\varepsilon.\]
        Примеры:    
        \begin{enumerate}
            \item $x_n = c \implies \lim x_n = c$
            \item $\frac{x^2}{x^2+1} \implies \lim x_n=1 \impliedby \left|x_n-1\right|=\left|\frac{n^2}{n^2+1}-1\right|=\left|\frac{1}{n^2+1}\right|<\varepsilon \iff  \frac{1}{\varepsilon} < n^2+1$ 
            \item $x_n=\left( -1 \right)^{n}$ - предела нет.
        \end{enumerate}
        \end{definition}
        \begin{dlemma}
            Если $n_0$ подходит для некоторого  $\varepsilon_0$, то оно подходит для всех $\varepsilon > \varepsilon_0$
        \end{dlemma}
        \begin{dlemma}
            Если из последовательности выкинуть или добавить конечное число элементов, то её предел не измениться.    
        \end{dlemma}
        \begin{definition}
            $A \subset \mathbb{R}$, $\ell \in \mathbb{R}$. $\ell$ - пределььная точка  $A$, если в любом интервале содержащем  $\ell$, бесконечно много точек  $A$.
        \end{definition}
        \subsection{Свойства последовательностей}
        \begin{theorem}[Единственность предела]
            Если предел существует, то он единственнен.
            \begin{proof}
                Пусть $\lim x_n =a$, и  $\lim x_n=b$,  $a<b$.
                Рассмотрим интервалы $\left( c_1; \frac{a+b}{2} \right) $ и $\left( \frac{a+b}{2}; c_2 \right) $. Вне каждого из них лежит конечно число элементов. Но так-как интервалы не пересекаются, вся последовательноть лежит вне этих интервалов. Тогда в последовательности было-бы конечное число членов.
            \end{proof}
        \end{theorem}
        \begin{tlemma}
            \label{mixer}
            Пусть $\lim x_n=a$,  $\lim y_n=b$, и  $\varepsilon>0$. Тогда
            \[ \exists{n_0 \in \mathbb{N}}\quad \forall{n>n_0}\quad 
                \begin{cases}
                    |x_n-a|<\varepsilon\\
                    |y_n-b|<\varepsilon
                \end{cases}
            .\]     
        \begin{proof}
            \[ \exists{n_1 \in \mathbb{N}}\quad \forall{n > n_1}\quad  |x_n-a|<\varepsilon.\]
            \[ \exists{n_2 \in \mathbb{N}}\quad \forall{n > n_2}\quad   |y_n-b|<\varepsilon .\]
            \[ n_0 = \max\left( n_1, n_2 \right)  .\qedhere\] 
        \end{proof}
        \end{tlemma}
        \begin{theorem}[Предельный переход в неравенстве]
            Пусть $\forall{n \in \mathbb{N}}\quad x_n\le y_n$, $\lim x_n = a$,  $\lim y_n = b$, тогда  $a \le  b$.
             \begin{proof}
                 Если $a>b$: возьмём  $\varepsilon=\frac{a-b}{2}$, тогда 
                 \[ \begin{cases}
                     |x_{n_0} - a| < \varepsilon \implies x_{n_0} \ge  a-\varepsilon\\
                     |y_{n_0} - b| < \varepsilon \implies y_{n_0}\le b+\varepsilon
                 \end{cases} .\] 
                 \[ a-\varepsilon \le x_{n_0} \le y_{n_0} \le b+\varepsilon \implies a-\frac{a-b}{2}<b-\frac{a-b}{2} .\]
                 Что невозможно.
             \end{proof}    
             Примечание: Строгие неравенства могут не сохраняться.
        \end{theorem}
        \begin{theorem}
            Если последовательность имеет предел, то она ограниченна.
            \begin{proof}
                Пусть $\ell=\lim x_n$, рассмотрим интервал $\left( \ell-1; \ell+1 \right) $. Больше $\ell+1$ только конечное число элементов, значит среди них есть наибольший. Симметрично для наименьшего.
            \end{proof}
        \end{theorem}
        \begin{theorem}[Стабилизация знака]
            Если $\ell=\lim x_n \neq 0$, то начиная с некоторого номера, все члены последовательности имеют тот-же знак что и $\ell$.
             \begin{proof}
                 Докажем для $\ell>0$. Тогда, начиная с некоторого номера все элементы лежат в  $\left( 0; 2\ell \right) $. Для $\ell<0$ симметрично.
            \end{proof}
        \end{theorem}
        \begin{theorem}[Теорема о сжатой последовательности (о двух милиционерах)]
            Пусть $\forall{n \in \mathbb{N}}\quad x_n \le y_n \le z_n$. $\ell = \lim x_n = \lim z_n \implies \lim y_n = \ell$.
            \begin{proof}
                Возьмём $\varepsilon>0$ и  $n$ из леммы \ref{mixer}. Тогда
                 \[ 
                    \begin{cases}
                        |x_n-\ell|<\varepsilon \implies x_n > \ell-\varepsilon\\
                        |z_n-\ell|<\varepsilon \implies z_n < \ell+\varepsilon
                    \end{cases}
                 \implies \ell - \varepsilon < x_n \le y_n \le z_n < \ell + \varepsilon \implies \lim y_n = \ell.\qedhere\]    
            \end{proof}
        \end{theorem}
        \begin{tlemma}
            Если $\lim z_n = 0$,  $\forall{n \in \mathbb{N}}\quad |y_n| < z_n$, то $\lim y_n = 0$. 
            \begin{proof}
                $-z_n \le y_n \le z_n \implies \lim y_n = 0$
            \end{proof}
        \end{tlemma}
        \begin{definition}
            Бесконечна малая последоавтельность - последовательно, предел которой равен нулю.
        \end{definition}
        \begin{dlemma}
            Если $x_n$ - бесконечна малая, а  $y_n$ - ограниченна, то  $x_ny_n$ - бесконечно малая.
             \begin{proof}
                 \[ |y_n| < M .\]
                 \[ \exists{\varepsilon > 0}\quad \exists{n_0}\quad \forall{n > n_0}\quad |x_n| < \frac{\varepsilon}{M} \implies |x_ny_n| < \varepsilon .\] 
            \end{proof}
        \end{dlemma}
        \begin{theorem}[Теорема об арифметических действиях с пределами]
            Пусть $\lim x_n=a$ и  $\lim y_n = b$. Тогда:
            \[ \lim \left( x_n+y_n \right) =a+b .\]
            \[ \lim \left( x_ny_n \right) =ab .\]
            \[ \lim\left( x_n-y_n \right)=a-b  .\]
            \[ y_n \neq 0\land b\neq 0 \implies \lim\left( \frac{x_n}{y_n} \right) = \frac{a}{b} .\]
            \begin{proof}
                Берём $\varepsilon > 0$, и по  $\frac{\varepsilon}{2}$ берём нормер из леммы \ref{mixer}.
                \[ 
                    \begin{cases}
                        |x_n-a|<\frac{\varepsilon}{2}\\
                        |y_n-b|<\frac{\varepsilon}{2}
                    \end{cases}
                \implies |x_n+y_n-a-b|\le |x-a|+|y-b|<\frac{\varepsilon}{2}+\frac{\varepsilon}{2}=\varepsilon.\]
            Рассмотрим разность $x_ny_n-ab = x_ny_n-x_nb+x_nb-ab \implies |x_ny_n-ab|\le |x_n| |y_n-b|+|b| |x_n-a|$.\\
            $\lim x_n = a \implies \exists{M}\quad \forall{n \in \mathbb{N}}\quad x_n \le  M$.
            \[ \hat{\varepsilon} = \frac{\varepsilon}{2\max\left( M, |b| \right) } \implies \exists{n_0 \in \mathbb{N}}\quad \forall{n > n_0}\quad \begin{cases}
                |x_n-a| < \hat{\varepsilon}\\
                |y_n-b| < \hat{\varepsilon}
            \end{cases}.\] 
            \end{proof} 
        \end{theorem}
\end{document} 
