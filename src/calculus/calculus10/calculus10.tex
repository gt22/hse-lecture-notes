% !TEX encoding = UTF-8 Unicode
\documentclass[11pt, oneside]{article}   	% use "amsart" instead of "article" for AMSLaTeX format
\usepackage{amssymb}
\usepackage{amsmath}
\usepackage{cRussian}
\usepackage{cPicture}
\usepackage{cTheorem}
\usepackage{accents}
%\usepackage{cTikz}
\title{Мат. Анализ 10}
\author{Igor Engel}
\date{}

\begin{document}
\maketitle
\section{}
    \subsection{}
    \[ \ln(1+x) = x - \frac{x^2}{2} + \frac{x^3}{3} - \frac{x^{4}}{4} + \ldots + (-1)^{n-1}\frac{x^{n}}{n} + o(x^{n}) .\] 
    \[ (1+x)^{p} = 1 + px + \binom{p}{2}x^2 + \binom{p}{3}x^3  + \ldots + \binom{p}{n}x^{n} + o(x^{n}).\]
    \begin{theorem}
        При всех $x\in \mathbb{R}$ :
        \[ e^{x} = \sum\limits_{n=0}^{\infty} \frac{x^{n}}{n!} .\] 
        \[ \cos x = \sum\limits_{n=0}^{\infty} (-1)^{n} \frac{x^{2n}}{(2n)!} .\]
        \[ \sin x = \sum\limits_{n=x}^{\infty} (-1)^{n} \frac{x^{2n+1}}{(2n+1)!} .\]
        \begin{proof}
            \[ \forall{t\in \mathbb{R}}\quad \forall{n}\quad \left|\sin^{(n)}(t)\right| < 1 .\]
            \[ \forall{t\in \mathbb{R}}\quad \forall{n}\quad \left|\cos^{(n)}(t)\right| < 1 .\]
            Тогда
            \[ \lim\limits_{n \to \infty} \sum\limits_{k=0}^{n} (-1)^{k}\frac{x^{2k}}{(2k)!} = \cos x .\]
            Аналогично для синуса.\\
            Для $e^{x}$ рассмотрим отрезко $\left[0; x\right]$ (или $\left[x; 0\right]$).\\
            Тогда на этом отрезок $\forall{t\in \left[0; x\right]}\quad \forall{n}\quad (e^{x})^{(n)} \le \max\left( e^{x}, 1 \right) $.\\
            Тогда на этом отрезке (включая точку $x$) функция равна своему ряду.
        \end{proof}
    \end{theorem}
    \begin{theorem}
        $e \not\in \mathbb{Q}$ 
        \begin{proof}
        Поедположим, что $e\in \mathbb{Q}$. Тогда $\exists{n,m\in \mathbb{N}}\quad \frac{m}{n}$\\
        Так-как $2 <e < 3$, то $n\ge 2$.\\
        Заметим, что 
        \[ \frac{m}{n} e = e^{1} = 1 + \frac{1}{1!} + \frac{1}{2!} + \ldots + \frac{1}{n!} + \frac{e^{c}}{(n+1)!} .\]
        При этом, $0 < c < 1$.\\
        \[ m(n-1)! = n! + \frac{n!}{1!} + \frac{n!}{2!} + \ldots + \frac{n!}{n!} + \frac{e^{c}}{n+1} .\]
        При этом, $m(n-1)!\in \mathbb{Z}$.\\
        Сумма отношений факториалов так-же целая.\\
        Тогда $\frac{e^{c}}{n+1}\in \mathbb{N}$.\\
        Тогда $\frac{e^{c}}{n+1} \ge 1$.\\
        Так-как $c<1$, то $e^{c} < e$, а т.к $n\ge 2$, то $n+1 \ge 3$, а $\frac{e}{3} < 1$.\\
        Значит, $e \not\in \mathbb{Q}$.
        \end{proof}
    \end{theorem}
    \section{Точки экстремума}
    \begin{definition}
       $f: (E \subset \mathbb{R}) \mapsto \mathbb{R}$, $a\in E$. \\
       $a$ - точка локального минимума, если $\exists{U_a}\quad \forall{x\in U_a\cap E}\quad f(a) \le f(x)$.\\
       $a$ - точка строго локального минимума, если $\exists{\accentset{\circ}{U}_a}\quad \forall{x\in \accentset{\circ}{U}\cap E}\quad f(a) < f(x)$.
       Аналогично для максимума.\\
    \end{definition}
    \begin{definition}
       $a$ - точка экстремума, если $a$ - точка локального максимума, или локального минимума.
    \end{definition}
    \begin{theorem}[Необходимое условие экстремума]
       $f: \left<a, b\right> \mapsto \mathbb{R}$, $x_0\in (a, b)$, $f$ дифференцируема в  $x_0$.\\
       Если $x_0$ - точка экстремума, то производная в этой точке равна нулю.\\
       \begin{proof}
           Для определённости - $x_0$ - локальный максимум.\\
           \[ \exists{\delta > 0}\quad \forall{x\in \left<a, b\right>}\quad x\in (x_0-\delta; x_0+\delta) \implies f(x) \le f(x_0) .\]
           Функция $\left.f\right|_{(x_0-\delta; x_0+\delta)}$ имеет глобальный максимум в точке $x_0$, и по теореме Ферма производная равна нулю.
       \end{proof}
    \end{theorem}
    \begin{tlemma}
        Обратное неверно.\\
        Например, у $x^{3}$ производная в нуле равна нулю, но экстремума в нуле нет.
    \end{tlemma}
    \begin{tlemma}
        Экстремум может быть в точке, в которой функция недифференцируема.
        Например, у $|x|$ нет производной в нуле, но есть минимум в нуле.
    \end{tlemma}
    \begin{theorem}[Достаточные условия экстремума в терминах первой производной]
        $f: \left<a, b\right> \mapsto \mathbb{R}$, $x_0\in (a; b)$, $f$ непрерывна в $x_0$ и дифференцируема в на $(x_0-\delta;x_0) \cup (x_0; x_0+\delta)$.\\
        Тогда, если
        $f'>0$ на $(x_0-\delta; x_0)$ и $f' < 0$ на $(x_0; x_0+\delta)$, то $x_0$ - строгий локальный максимум.\\
        Нестрогие занки дуют нестрогий максимум, обратные знаки - минимум.
        Если $f'$ не меняет знак, то $x_0$ не экстремум.
        \begin{proof}
            Пусть для определённости $f'>0$ на $(x_0-\delta; x_0)$ и $f'<0$ на $(x_0; x_0+\delta)$.\\
            Рассмотрим полуинтервал $(x_0-\delta; x_0]$, $f$ на нём непрерывна, и $f'>0$ в интервале. Значит, $f$ строго возрастает на полуинтервале, и $\forall{x_0-\delta; x_0}\quad f(x) < f(x_0)$.\\
            Аналогично на полуинтервале $[x_0; x_0+\delta)$, на нём $f$ строго убывает. $\forall{x\in (x_0; x+\delta)}\quad f(x) < f(x_0)$.\\
            Совмещая, получаем:
            \[ \forall{x\in (x_0-\delta; x_0)\cup (x_0; x_0+\delta)}\quad f(x) < f(x_0)  .\]
            А значит $x_0$ - строгий локальный максимум. Остальные доказываются аналогично.
        \end{proof}
    \end{theorem}
    \begin{theorem}[Достаточные условия экстремума в термниах второй производной]
        $f: \left<a, b\right> \mapsto \mathbb{R}$, $x_0\in (a, b)$, $f$ дважды дифференцируема в $x_0$, $f'(x_0) = 0$.\\
        Тогда
        \begin{enumerate}
            \item $f''(x_0) > 0$ - $x_0$ - точка строго локального минимума.
            \item $f''(x_0) < 0$ - $x_0$ - точка строго локального максимума.
        \end{enumerate}
        Доказывается как частный случай следующей теоремы.
    \end{theorem}
    \begin{theorem}[Достаточные условия экстремума в термниах $n$-й производной]
        $f: \left<a, b\right> \mapsto \mathbb{R}$, $x_0\in (a, b)$, $f$ $n$ раз дифференцируема в $x_0$, $\forall{k\in [1; n)}\quad f^{(k)} = 0$.\\
        Тогда
        \begin{enumerate}
            \item $n$ чётно, и $f^{(n)}(x_0) > 0$ - $x_0$ - точка строго локального минимума.
            \item $n$ чётно, и $f^{(n)}(x_0) < 0$ - $x_0$ - точка строго локального максимума.
            \item $n$ нечётно, и $f^{(n)}(x_0) \neq  0$, то $x_0$ - не точка экстремума.
        \end{enumerate}
        \begin{proof}
            \[ f(x) = f(x_0) + \frac{f^{(n)}(x_0)}{n!}(x-x_0)^{n} + o((x-x_0)^{n}) .\]
            \[ f(x) - f(x_0) = (x-x_0)^{n}\left( \frac{f^{(n)}(x_0)}{n!} + o(1) \right)  .\] 
            Рассмотрим первый случай:\\
            $n$ чётно, $(x-x_0)^{n}>0$, знак определяется вторым множетелем, котрый стремится к $\frac{f^{(n)}(x_0)}{n!}>0$. По теореме о стабилизации знака, $f(x)-f(x_0)>0$ в некоторой окрестности $x_0$. Значит, $x_0$ является локальным минимумом.\\
            Второй случай аналогично.\\
            Третий случай:\\
            $n$ нечётно, значит $(x-x_0)^{n} < 0$ если $x<x_0$, и $(x-x_0)^{n}<0$ если $x>x_0$.\\
            Пусть $f^{(n)}(x_0)$ положительна, тогда функция $f(x)-f(x_0)$ поменяет знак в точке $x_0$, и $x_0$ не может быть экстремумом. Аналогично для отрицательной производной.
        \end{proof}
    \end{theorem}
\section{Выпуклые функции}
    \begin{definition}
        Функция $f: \left<a, b\right> \mapsto \mathbb{R}$ называется выпуклой, если:
        \[ \forall{x, y\in \left<a, b\right>}\quad \forall{\lambda\in (0; 1)}\quad f(\lambda x + (1-\lambda)y) \le \lambda f(x) + (1-\lambda)f(y) .\]
        Если знак строкий и $x\neq y$, то $f$ называется строго выпуклой.\\
        $f$ называется вогнутой, если 
        \[ \forall{x,y\in \left<a, b\right>}\quad \forall{\lambda\in (0; 1)}\quad f(\lambda x + (1-\lambda)y) \ge \lambda f(x) + (1-\lambda)f(y) .\]

    \end{definition}
\end{document} 
